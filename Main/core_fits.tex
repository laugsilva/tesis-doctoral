%
%%%%%%%%%%%%%%%%%%%%%%%%%%%%%%%%%%%%%%%%%%%%%%%%%%%%%%%%%%%%%%%%%%%%%%%%%%%%%%%
% Fractio of gluon splitting in data
%%%%%%%%%%%%%%%%%%%%%%%%%%%%%%%%%%%%%%%%%%%%%%%%%%%%%%%%%%%%%%%%%%%%%%%%%%%%%%%
%
\chapter{Fraction of double $b$-hadron jets in data}\label{ch:gbbfraction}

%------------------------------------------------------------------------
\section{Maximum likelihood fits}\label{sec:LLFits}
%------------------------------------------------------------------------

Maximum likelihood (ML) information only parametrizes the shape of a 
distribution (i.e. one can determine fraction of signal events from 
MC fits but no number of signal events).

The extended version of the maximum likelihood approach adds an extra term
allowing the estimation of the absolute number of signal/background events.

%Adding pdfs
For N p.d.f.s, there are N-1 fraction coefficients that should sum to less 1. The remainder is by construction 1 minus the sum of all other coefficients.


%Binned or unbinned MF fit?
%Roofit presentation page 112
%In most RooFit applications it doesn't matter. Internally binned data is represented the same way as unbinned data, a ROOT TTree with the bin coordinates.
%WEIGHTS ARE SUPPORTED IN UNBINNED DATASETS!! But use with care. Error analysis in ML fits to weighted unbinned data can be complicated...

\subsubsection{Unbinned fits}


\subsubsection{Fitting and likelihood minimization}
What happens when you do pdf->fitTo(*data)?

- Construct object representing -log of (extended) likelihood

- Minimize likelihood w.r.t floating parameters using MINUIT [REFERENCIAS].




%------------------------------------------------------------------------
\section{Results}\label{sec:FitsResults}
%------------------------------------------------------------------------




-TODO: TRUE FRACTIONS IN PYTHIA

DECIR QUE HICIMOS LOS FITS PARA 5 Y 3 PARAMETROS. MOSTRAR LOS RESULTADOS Y EXPLICAR QUE ES DIFICIL DISTINGUIR ENTRE LOS TEMPLATES DE SINGLE-B Y SINGLE-C Y QUE POR ESO VAMOS A HACER TODO FIJANDO LAS FRACCIONES DE SINGLE-C Y MERGED-C A SINGLE-B Y MERGED-B RESPECTIVAMENTE.

PARA JUSTIFICAR QUE HACEMOS LOS FITS CON 3 PARAMETROS PODEMOS MOSTRAR LOS HISTOS Y DECIR QUE LAS CORRELACIONES SON MUY GRANDES (PONER NUMEROS). EN ALGUN MOMENTO TENDRA QUE HABER UN C-TAGGER.  EN DIJET TENEMOS SINGLE-C'S, EN TTbar NO HAY C'S, SI HAY MERGED-C PERO ES ALGO QUE QUEREMOS REMOVER DE TODAS MANERAS.



COMO RESULTADO PRESENTAREMOS ENTONCES LAS FRACCIONES DE SINGLE, MERGED Y LIGHT.



\begin{table}[!hbt] %[h]
\renewcommand{\arraystretch}{1.2}
\centering
\begin{tabular}{ | c || c | c || c | c || c | c ||}
  \hline
  Jet $\pt$ & \multicolumn{2}{c||}{single $b$-jet} & \multicolumn{2}{c||}{merged $b$-jet} & \multicolumn{2}{c||}{~light jet~}\\ \cline{2-7}
    (GeV ) & fit result & ~stat.err. & fit result & ~stat.err. & fit result & ~stat.err.\\ \hline
   40 - 60 &  62\% &  3\%  &  ~~3\%  &  ~~1\% &  ~4\%  &  4\%   \\ 
   60 - 80 &  62\% &  1\%  &  5.2\%  &  0.4\% &  ~2\%  &  2\%   \\ 
   80 - 110&  57\% &  1\%  &  8.5\%  &  0.4\% &  ~3\%  &  2\%   \\ 
  110 - 150&  55\% &  2\%  &  ~13\%  &  ~~1\% &  ~1\%  &  4\%   \\ 
  150 - 200&  53\% &  3\%  &  ~15\%  &  ~~1\% &  ~0\%  &  4\%   \\ 
  200 - 270&  53\% &  5\%  &  ~17\%  &  ~~1\% &  -1\%  &  7\%   \\ 
  270 - 360&  48\% &  3\%  &  ~19\%  &  ~~1\% &  ~4\%  &  4\%   \\ 
  360 - 480&  39\% &  5\%  &  ~21\%  &  ~~1\% &  15\%  &  6\%   \\ \hline
\end{tabular}
\caption{Measured fractions of single, merged and light $b$-tagged jets in experimental data from 2011 run.}
\label{tb:fitfractions}
\end{table}







%------------------------------------------------------------------------
\section{Systematic uncertainties}\label{sec:FractionSystematics}
%------------------------------------------------------------------------

- TO DO: HACER JER SI PUEDO


In order to study the systematic uncertainties in the results the following contributions were evaluated:

\begin{itemize}\addtolength{\itemsep}{-0.4\baselineskip}
\item
uncertainty in the track reconstruction efficiency;
\item
uncertainty in the jet transverse momentum resolution;
\item
uncertainty in the jet energy scale.
%the effect of jet isolation
%\item
%other $\Delta R$ cuts for B-labeling and matching
\end{itemize}


\vspace{3mm}
The different contributions to the systematic uncertainty are summarized in Table~\ref{tb:systematicsfits}.
\begin{table}[!hbt] %[h]
\renewcommand{\arraystretch}{1.2}
\centering
\begin{tabular}{ | c | c |}
\hline
  ~~~~~~~Systematic source~~~~~~~ &~~Uncertainty~~\\ \hline
  track reconstruction efficiency  &    negligible\%        \\ 
  jet $\pt$ resolution  &    2\%        \\  
  jet energy scale  &    2\%        \\ \hline 
\end{tabular}
\caption{Systematic uncertainties.}
\label{tb:systematicsfits}
\end{table}





- CAMBIAR LAS FRACCIONES C/B EN 20\% Y VER CUANTO CAMBIAN LOS RESULTADOS. LA FRACCION DE C+B NO DEBERIA CAMBIAR:  EL NUMERO TOTAL DE MERGED-C+MERGED-B NO CAMBIA POR VARIAR LA FRACCION DE C/B. EN REALIDAD ESTAMOS MIDIENDO LA FRACCION DE SINGLE B+C Y LA FRACCION DE MERGED-C + B.

%------------------------------------------------------------------------
\section{Enriched samples in single and merged $b$-jets}\label{sec:Enriched}
%------------------------------------------------------------------------

MOSTRAMOS QUE LOS MISMOS TEMPLATES USADOS EN OTRA MUESTRA FUNCIONAN BIEN Y QUE EL RESULTADO ESTA EN ACUERDO CON LA PREDICCION DE PYTHIA (ESTO NO ES OBVIO PARA NADA).
(PODRIAMOS DECIR QUE LA FRACCION DE FLAVOR CREATION ESTA BIEN SIMULADA POR PYTHIA)

\begin{table}[!hbt] %[h]
\renewcommand{\arraystretch}{1.2}
\centering
\begin{tabular}{ | c || c | c | c ||}
  \hline
  Jet $\pt$ & \multicolumn{3}{c||}{single $b$-jet}\\ \cline{2-4}
    (GeV ) & ~~~~fit result~~~ & ~~~~stat.err.~~~~ & pythia prediction \\ \hline
   40 - 60 &  99\%  &  11\%  &  84\% \\  
   60 - 80 &  82\%  &  ~5\%  &  87\% \\ 
   80 - 110&  84\%  &  ~5\%  &  88\% \\ 
  110 - 150&  86\%  &  ~8\%  &  85\% \\ 
  150 - 200&  89\%  &  ~9\%  &  83\% \\ 
  200 - 270&  95\%  &  15\%  &  80\% \\ 
  270 - 360&  67\%  &  11\%  &  81\% \\ 
  360 - 480&  73\%  &  16\%  &  73\% \\ \hline
\end{tabular}
\caption{Measured fractions of single $b$-jets in experimental data from 2011 run, enriched in single $b$-jets.}
\label{tb:fitfractions2btagS}
\end{table}

\begin{table}[!hbt] %[h]
\renewcommand{\arraystretch}{1.2}
\centering
\begin{tabular}{ | c || c | c | c ||}
  \hline
  Jet $\pt$ & \multicolumn{3}{c||}{merged $b$-jet}\\ \cline{2-4}
    (GeV ) & ~~~~fit result~~~ & ~~~~stat.err.~~~~ & pythia prediction \\ \hline
   40 - 60 &  -1\%  &  1\%  &  1\% \\  
   60 - 80 &  -3\%  &  1\%  &  1\% \\ 
   80 - 110&  ~2\%  &  1\%  &  1\% \\ 
  110 - 150&  ~4\%  &  2\%  &  3\% \\ 
  150 - 200&  ~4\%  &  2\%  &  3\% \\ 
  200 - 270&  ~7\%  &  2\%  &  5\% \\ 
  270 - 360&  12\%  &  2\%  &  6\% \\ 
  360 - 480&  10\%  &  1\%  &  8\% \\ \hline
\end{tabular}
\caption{Measured fractions of merged $b$-jets in experimental data from 2011 run, enriched in single $b$-jets.}
\label{tb:fitfractions2btagM}
\end{table}

\begin{table}[!hbt] %[h]
\renewcommand{\arraystretch}{1.2}
\centering
\begin{tabular}{ | c || c | c | c ||}
  \hline
  Jet $\pt$ & \multicolumn{3}{c||}{light $b$-jet}\\ \cline{2-4}
    (GeV ) & ~~~~fit result~~~ & ~~~~stat.err.~~~~ & pythia prediction \\ \hline
   40 - 60 &  ~-7\%  &  11\%  &  5\% \\  
   60 - 80 &  ~17\%  &  ~6\%  &  2\% \\ 
   80 - 110&  ~~4\%  &  ~6\%  &  1\% \\ 
  110 - 150&  ~-1\%  &  ~9\%  &  1\% \\ 
  150 - 200&  ~-6\%  &  10\%  &  2\% \\ 
  200 - 270&  -17\%  &  17\%  &  3\% \\ 
  270 - 360&  ~~9\%  &  11\%  &  4\% \\ 
  360 - 480&  ~~4\%  &  16\%  &  8\% \\ \hline
\end{tabular}
\caption{Measured fractions of light $b$-jets in experimental data from 2011 run, enriched in single $b$-jets.}
\label{tb:fitfractions2btagL}
\end{table}
