%
%%%%%%%%%%%%%%%%%%%%%%%%%%%%%%%%%%%%%%%%%%%%%%%%%%%%%%%%%%%%%%%%%%%%%%%%%%%%%%%
% Fractio of gluon splitting in data
%%%%%%%%%%%%%%%%%%%%%%%%%%%%%%%%%%%%%%%%%%%%%%%%%%%%%%%%%%%%%%%%%%%%%%%%%%%%%%%
%
\chapter{Fraction of double $b$-hadron jets in data}\label{ch:gbbfraction}

%------------------------------------------------------------------------
\section{Maximum likelihood fits}%\label{sec:FractionSystematics}
%------------------------------------------------------------------------

Maximum likelihood (ML) information only parametrizes the shape of a 
distribution (i.e. one can determine fraction of signal events from 
MC fits but no number of signal events).

The extended version of the maximum likelihood approach adds an extra term
allowing the estimation of the absolute number of signal/background events.

%Adding pdfs
For N p.d.f.s, there are N-1 fraction coefficients that should sum to less 1. The remainder is by construction 1 minus the sum of all other coefficients.


%Binned or unbinned MF fit?
%Roofit presentation page 112
%In most RooFit applications it doesn't matter. Internally binned data is represented the same way as unbinned data, a ROOT TTree with the bin coordinates.
%WEIGHTS ARE SUPPORTED IN UNBINNED DATASETS!! But use with care. Error analysis in ML fits to weighted unbinned data can be complicated...



\subsubsection{Fitting and likelihood minimization}
What happens when you do pdf->fitTo(*data)?

- Construct object representing -log of (extended) likelihood

- Minimize likelihood w.r.t floating parameters using MINUIT [REFERENCIAS].


%------------------------------------------------------------------------
%\section{Systematic uncertainties}\label{sec:FractionSystematics}
%------------------------------------------------------------------------
