%%%%%%%%%%%%%%%%%%%%%%%%%%%%%%%%%%%%%%%%%%%%%%%%%%%%%%%%%%%%%%%%%%%%%%%%%%%%%%%
%\chapter{Kinematic differences between single and double $B$-hadron jets}
\chapter{Double $b$-hadron jet identification}\label{ch:kinematic}
%%%%%%%%%%%%%%%%%%%%%%%%%%%%%%%%%%%%%%%%%%%%%%%%%%%%%%%%%%%%%%%%%%%%%%%%%%%%%%%%

In this chapter we focus on the understanding of the internal structure of $b$-jets containing two $B$-hadrons by investigating the differences between those and single $b$-quark jets.  These differences %between single and double $b$-hadron or ``merged'' jets 
are expected to arise from the two-subjet (two $b$-hadrons) structure of double $b$-hadron or ``merged'' jets, which would tend to be wider and with a larger number of constituents. 
Based on these envisaged characteristics, simulated QCD samples of $b$-tagged jets were used to explore properties with potential discrimination power.  The Monte Carlo distributions were in turn compared to data from the 2011 run for validation.
We present results from these studies and discuss the choice of the observables selected to build the multivariable tool presented in Chapter~\ref{ch:mva}.


%------------------------------------------------------------------------
\section{Data samples and event selection}\label{sec:analysis}
%------------------------------------------------------------------------


The tagging technique presented in this thesis relies on Monte Carlo predictions for the signal (single $b$) or background (merged $b$) hypotheses. The accuracy of the simulation is validated with data by comparing the distributions of the different variables studied.

The data samples employed correspond to proton-proton collisions at $\sqrt{s}=7$ TeV delivered by the LHC and recorded by ATLAS between May and November 2011, with the LHC running with 50~ns bunch spacing, and bunches organized in bunch trains. Only data collected during stable beam periods in which all sub-detectors were fully operational are used. After the application of the data quality selection, the  surviving data corresponds to an integrated luminosity of 4.7~fb$^{-1}$. The LHC instantaneous luminosity steadily increased during 2011. As a result, the average number of minimum-bias pile-up events, originating from collisions of additional protons in the same bunch as the signal collision, grew from from 3 to 20. This fact will be of importance when discussing the selection of discriminating variables.  

The Monte Carlo event generators discused in Section~\ref{sec:MCtools} are used here. Samples of dijet events from proton-proton collision processes were simulated with {\sc Pythia} version 6.423~\cite{PYTHIA6}, used both for the simulation of the hard $2\rightarrow 2$ process as well as for the parton shower, underlying event, and hadronization models. The ATLAS AMBT2 tune of the soft model parameters was used~\cite{Pythia_MC11tune}.
In order to have sufficient statistics over the entire $\pt$ spectrum, eight samples were generated with different thresholds of the hard-scattering partonic transverse momentum $\hat{p}_T$. Events from different samples were mixed taking into account their respective production cross sections.
The simulated data sample used for the analysis %(MC11b)
gives an accurate description of the pile-up content and detector conditions for the full 2011 data-taking period. 



%------------------------------------------------------------------------
%\subsection{Event selection}\label{sec:EventSelection}
%------------------------------------------------------------------------

%\vspace{5 mm}

The event selection and quality criterion used to extract, from the data and Monte Carlo samples, the final set of jets for the analysis comprises different steps: 

\vspace{5 mm}
\emph{Trigger}. The event sample was collected  using the ATLAS single jet triggers which select events with at least one jet with transverse energy above a given threshold.  At the hardware Level 1 and local software Level 2 (see Section~\ref{sec:TDAQ}), cluster-based jet triggers are used to select events with high-$pt$ jets. The Event Filter, in turn, runs  the offline anti-$k_t$ jet finding algorithm with $R = 0.4$ on topological clusters over the complete calorimeter.  At this stage, the transverse energy thresholds, expressed in GeV, are: 20, 30, 40, 55, 75, 100, 135, 180. These triggers reach an efficiency of 99\% for events having the leading jet with an offline energy higher than the corresponding trigger thresholds by a factor ranging between 1.5 and 2. The jet triggers with the lowest $\pt$ thresholds were prescaled by up to five orders of magnitude, and typically the same jet trigger is prescaled ten times more in the later data taking periods compared to the early ones. 

\vspace{5 mm}
\emph{Primary vertex}.  The offline event selection requires at least one primary vertex candidate with 5 or more tracks.  No requirements are placed on the longitudinal position (along the beam line) of the vertex as the beam spot is used as a constraint when fitting the vertex. 

\vspace{5 mm}
\emph{Primary jet algorithm}. The jet algorithm selected for the analysis was the ATLAS default anti-$k_t$ algorithm~\cite{antiktalg}, with a distance parameter $R = 0.4$, using calorimeter topological clusters~\cite{topoClusters} as input.

\vspace{5 mm}
\emph{Jet calibration}.  The EM+JES calibration scheme, described in Section\ref{sec:calib}, was used to correct the jet energies for inhomogeneities and for the non-compensating nature of the calorimeter.

\vspace{5 mm}
\emph{Jet quality}.  Several quality criteria are applied to eliminate ``fake'' jets that are caused by noise bursts in the calorimeters and energy depositions belonging to a previous bunch crossing~\cite{ATLAS-CONF-2012-020}.

\vspace{5 mm}
\emph{Jet tagging}.  Only jets tagged as $b$-jets using the MV1 $b$-tagging algorithm at the 60\% efficiency working point were considered.

\vspace{5 mm}
\emph{Isolation}.  $b$-tagged jets with close-by jets ($\Delta R < 0.8$) with $\pt$ higher than 7~GeV at electromagnetic scale were not included in the analysis.

\vspace{5 mm}
All jets, with transverse momentum between 40 and 480~GeV,  the selected $\pt$ range for the analysis, were required to be in a region with full tracking coverage, $|\eta_{jet}|<2.1$, and they were classified in eight $\pt$ bins chosen such as to match the jet trigger 99\% efficiency thresholds (in GeV): 40, 60, 80, 110, 150, 200, 270, 360. An event is used if it satisfies the highest threshold trigger that is 99\% efficient for the $\pt$ bin that corresponds to the $\pt$ of its leading jet.

In the case of MC, the reconstructed $b$-tagged jets were further classified into single and merged $b$-jets based on truth Monte Carlo information. A $b$-hadron is considered to be associated to a jet if the $\Delta R$ distance in $\eta-\phi$ space between the direction of the hadron and the jet axis is smaller than 0.4. Jets were labeled as merged (single) $b$-jets if they contain two (only one) $b$-hadron:%\footnote[2]{Another approach is to match particles using the 'active area' of the jet, defined with the concept of ghost-particles but utilizing true particles as ghost constituents\cite{CatchmentArea}. This procedure will be implemented in a future version of this analysis.}.
\begin{equation}
\mbox{single $b$-jets:} \; \; \Delta R(j,b_{1/2}) < 0.4
\label{eqn:single}
\end{equation}
\begin{equation}
\mbox{merged $b$-jets:}  \; \; \Delta R(j,b_1) < 0.4 \; \& \;  \Delta R(j,b_2) < 0.4
\label{eqn:merged}
\end{equation}
%
where $j$ is a jet in the event and $b_{1/2}$ are the $b$-hadrons in the event. In the case another size parameter is used for jet finding, the definitions in equations~\ref{eqn:single} and~\ref{eqn:merged} change accordingly.


%------------------------------------------------------------------------
\subsection{Track selection }\label{sec:TrackSelection}
%------------------------------------------------------------------------

It is important to select genuine tracks belonging to jets. Only tracks located  within a cone of radius $\Delta R(j,\mbox{track}) \leq 0.4$ around the jet axis were considered. %\footnote[2]{In a future version of the study the concept of ghost-particles\cite{CatchmentArea} can be applied in the track-to-jet assignment using tracks as ghost constituents.}.
  Cuts on $\pt^{\mbox{trk}}>1.0$~GeV and the $\chi^2$ of the track fit, $\chi^2/{\it ndf}<3$, are applied. % as a minimum starting point. 
 In addition, tracks are required to have a total of at least seven precision hits (pixel or micro-strip) in order to guarantee at least 3 $z$-measurements. Tracks are also required to fulfill cuts on the transverse and longitudinal impact parameters at the perigee to ensure that they arise from  the primary vertex. As cutting on impact parameter (IP) significance might be detrimental for $b$-jets, where large IP values are expected, relaxed cuts were used, $|IP_{xy}|<2$~mm, and $|IP_{z}\sin\theta|<2$~mm, with $\theta$ being the polar angle measured with respect to the beam axis. The track quality cuts are summarized in table~\ref{tb:tracks}. %This, however, might still be too tight for $b$-jets.


\begin{table}[!hbt] %[h]
\renewcommand{\arraystretch}{1.2}
\centering
\begin{tabular}{ c  c  }
  \hline
  Track parameter &  Selection \\ \hline
  $\pt$   &   > 1~GeV \\
  $d_0^{PV}$   &   < 2~mm \\
  $z_0^{PV}\sin \theta$   &   < 2~mm \\
  $\chi^2 /ndof$   &   < 3 \\
  Number of Pixel hits   &  $\geq$ 2 \\
  Number of SCT hits   &   $\geq$ 4 \\
  Number of Pixel+SCT hits   &  $\geq$ 7 \\ \hline
\end{tabular}
\caption{Track selection criteria used for double $b$-hadron jet tagging, where $d_0^{PV}$ and $z_0^{PV}$ denote the transverse and longitudinal impact parameters derived with respect to the primary vertex. The $chi^2 / ndof$ is that of the track fit.}
\label{tb:tracks}
\end{table}




%------------------------------------------------------------------------
\section{Kinematic differences between single and double $b$-hadron jets}\label{sec:gbbKine} %$b$- and merged $b\bar{b}$-jets}\label{sec:gbbKine}
%\section{Full ATLAS Monte Carlo Analysis}\label{sec:gbbKine}
%------------------------------------------------------------------------ 


%------------------------------------------------------------------------
%PYTHIA STANDALONE
%------------------------------------------------------------------------


%See email from Jesse Thaler
%Thu, Jun 2, 2011 at 7:43 PM
%subject:	 Re: N-subjettiness Code
%This is really fascinating, and it is starting to make some physical sense.  
%For the single $b$-jets, $\tau_1$, $\tau_2$, and $\Delta R$ between the $k_T$ axes in the jet are all small which is expected for a pencil-like jet.  For the $b \bar{b}$-jets, these variables are all large, which is typical of a gluon jet.  But the correlations are really fascinating in merged $b$-jets.   $\tau_1$ and $\Delta R$ between the $k_T$ axes are nearly linearly related, which is expected if there are two hard lobes of energy.  But $\tau_2$ is almost independent of $\Delta R$ between the $k_T$ axes, meaning that regardless of where the axes are, the energy is uniformly distributed around them.
%So the question is whether you can make use of this.  tau1 and deltaR_12 are clearly useful variables, but they are also quite correlated.  tau2 appear to be uncorrelated with deltaR_12, but semi-correlated with tau_1.  By eye, the best discriminator for the R = 0.4 jets looks to be something like
%tau2 > (10 GeV/pT), deltaR_12 > (10 GeV/pT)
%or maybe
%(tau2 + deltaR_12) >  (20 GeV/pT).
%At this point, what would be helpful is to know whether your selection is really picking out g>bb jets or just picking up gluon jets in general.  For example, is the tau2 vs deltaR_12 non-correlation the same for generic gluon jets, or is it special to g>bb?  My intuition is that this must be a special feature of g>bb, since otherwise, it would be quite easy to separate gluon jets from quark jets...


%See this page for comparisons between g/b/bb from Max
%http://slac.stanford.edu/~swiatlow/gbb_plots/plots.html

%See mail from ariel 2 Jun 2011
%Quiza lo que este ocurriendo es que gbb fragmenta como normal (gluon) qcd, con mas splittings que b-jets (quarks) y sin la 2-body decay structura que estamos esperando.

%Date: Thu, 2 Jun 2011 20:37:57 +0200
%From: Ariel Schwartzman <sch@slac.stanford.edu>
%To: Maria Laura Gonzalez Silva <laugs@mail.cern.ch>
%Cc: Ricardo Piegaia <aia@df.uba.ar>, Laura <laugs@cern.ch>
%Subject: Re: N-subjettiness
% Muy interesante Laura.
% Tau1 aumenta con DRbb, como se espera, pues es como el jet width.
% Tau2 es casi flat con DR, lo que puede indicar dos cosas:
%  i) los kt-axis a los que se refiere Jesse no estan encontrando los dos B's
% ii) la contribucion de los tracks from B's es pequenia comparada con el resto de la fragmentacion, de modo de a que un gbb jet seria ungluon jet plus algunos soft displaced tracks on top.
% Te propngo algunos plots:
% 1) Jesse sugiere plotar el DR entre los dos kt axes para b y bb. Espero que quede claro en su codigo como axeder a esta variable...
% 2) Hace un 2D plot con la correlacion entre el DR entre los axes (1) y DR(B,B) para ver si esta definicion de axes corresponde a lo queesperamos.
%3) Vos habias mirado a DR(1,2) que es el DR entre los dos leading tracks. Podrias re-vivir estos estudios ahora con el generator study? Yhacer tambien el 2D plot de DR(1,2) vs DR(B,B)?
%4) Seria bueno repetir todas las input variables para pure gluon jets (esto lo sugeri en un mail el otro dia) para entender mejor cual es la diferencia entre gluones y gbb.
%5) Los dos blobs de energy que esperamos vienen de la hadronizacion de los B hadrons, mas que de la fragmentacion del gluon. Quiza estesea un key point que estamos ignorando. Yo estoy tentado a sugerir que calculemos las variables usando "displaced tracks" solamente. Ricardo, es posible ponerle un flag a las particulas que vienen de los B decays como para que Laura solo use estas particulas para calcular N-subjettiness?





The differences between genuine $b$-quark jets and $b \bar{b}$ jets are expected to arise from the two-subjet (two $B$-hadrons) substructure of merged jets.  They are thus expected, for the same jet $\pt$, to have higher track-multiplicity and be wider than single $b$-jets. Based on these characteristics %, the following properties were studied in simulated QCD samples of $b$-tagged jets using either calorimeter or track constituents:
simulated QCD samples of $b$-tagged jets were used to study the following properties, discussed in the next paragraphs, built from jet constituents either at calorimeter level (topological clusters) or tracks associated to the jet:

\begin{itemize}\addtolength{\itemsep}{-0.4\baselineskip}
\item
Jet multiplicity (number of constituents)
\item
Jet width, $\pt$ weighted %Track jet width ($\pt$ weighted)
\item 
Jet Mass
\item
Nr.\ of $k_t$ subjets %track-jets
\item
Maximum $\Delta R$ between pairs of constituents % (tracks)
\item
$\Delta R$ between 2 $k_t$ subjets within the $b$-jet
\item
$\tau_2$: 2-subjettiness 
\item
$\tau_2/\tau_1$
\item
$\Delta R$ of leading constituents %tracks
\item 
Eccentricity %(track & calo)
\end{itemize}



{ \em I. Jet track multiplicity}
\\[3mm]
This variable is defined as the number of tracks associated to the jet, it is simple to calculate and carries important information of the jet inner structure. Figure~\ref{fig:ntrksinglemerged} shows the distribution of the observable for single and merged $b$-jets.  It was observed that merged $b$-jets contain on average around two more tracks than single $b$-jets at low jet $\pt$, with a larger difference at higher $\pt$ values. The jet track multiplicity corresponds to tracks with $\pt$ above 1 GeV, satisfying the quality cuts described in section~\ref{sec:EventSelection}. The effect of using a minimum track $\pt$ of 0.5 GeV was also examined. This was motivated by the fact that it could lead to an improvement in discrimination if it captured more information about the fragmentation process.  On ther other hand, a lower minimum track $\pt$ can make the method more sensitive to pile-up with the addition of soft tracks incorrectly associated to the jets.  What it was observed is that reducing the $\pt$ cut only widens the distributions without increasing the separation between single and merged jets. 
\begin{figure}[tp]
\centering
\includegraphics[width=0.49\textwidth]{FIGS/VarsSingleMerged/Ntrk080.pdf}
\includegraphics[width=0.49\textwidth]{FIGS/VarsSingleMerged/Ntrk200.pdf}
\caption{Distribution of the track multiplicity in jets for single and merged $b$-jets between 80~GeV to 110~GeV (left) and 200~GeV to 270~GeV (right).}
\label{fig:ntrksinglemerged}
\end{figure}

{ \em II. Jet width}
\\[3mm]

%%%%%%%%%%%%%%%%%%%%%%%%%%%%%%%%%%%%%%%%%%%%%%%%%%%%%%%%%%%%%%%%%%%%%%%%%%%%%%%%%%%%%%%%%%%%%%%%%%
The jet width is part of a set of continuous variables that try to distinguish individual particles/subjets within the jet as a smooth funcion of $(\delta \eta, \delta \phi)$ away from the jet axis, in order to form combinations like geometric moments.  This particular combination sums the distances between the jet constituents and its axes, weighted by the constituent $\pt$, and then normalized to the total $\pt$ of the jet. The compact definition is 
%
\begin{equation} 
\mbox{ {\it Jet width}} = \frac{\sum_{i=1}^N \pt^{const_i} \,\Delta R (const_i,jet) }{\sum_{i=1}^N \pt^{const_i} }
\label{eqn:trackjetwidth}
\end{equation} 
%
where $N$ is the total number of calorimeter or track constituents.  This observable is also highly correlated to the mass of the jet.

This linear radial moment is a measure of the width or ``girth''~\cite{PhysRevLett.105.022001} of the jet.  Under the assumption of central jets with massless constituents at small angles, this linear moment is identical to jet broadening~\cite{Catani1992269}, defined as the sum of momenta transverse to the jet axis normalized by the sum of momenta. While jet broadening is natural at an $e^+ e^-$ collider, the linear radial moment is more natural at the LHC.

An alternative approach to measuring the width is to use the angular separation of the two hardest constituents inside jets. This has the advantage of effectively removing any dependence on the shower development within the calorimeter and focuses on the hard component of the jet.



%%%%%%%%%%%%%%%%%%%%%%%%%%%%%%%%%%%%%%%%%%%%%%%%%%%%%%%%%%%%%%%%%%%%%%%%%%%%%%%%%%%%%%%%%%%%%%%%%%

Figure~\ref{fig:trkwidthsinglemerged} shows the distribution for the Track-jet width. As expected, merged $b$-jets are wider than single $b$-jets. In Fig.~\ref{fig:ntrktrkwidthsinglemerged} the correlation between the track-jet width and the jet track multiplicity is shown for single and merged $b$-jets. These two variables alone provide a good discrimination for tagging $b \bar{b}$ jets.

The calorimeter jet width ( using topological clusters) gives also good separation. However, this variable is more sensitive to the amount of pile-up in the event than its track-based counterpart. In Fig.~\ref{fig:calowidthpileup} the distributions of calorimeter width for single and merged $b$-jets  can be seen for events with low and high Number of Primary Vertices (NPV), in a low $\pt$ region where the effect of pile-up is more important. In Fig.~\ref{fig:trkwidthpileup} the same distributions are shown for the track-jet width. Calorimeter jet width varies %significatively
with NPV and due to this behavior the track-based version is more suitable as a more robust discriminator. For similar reasons, the jet topological cluster multiplicity and the jet mass were discarded as discriminating variables.
\\[3mm]

\begin{figure}[tp]
\centering
\includegraphics[width=0.49\textwidth]{FIGS/VarsSingleMerged/trkWidth080.pdf}
\includegraphics[width=0.49\textwidth]{FIGS/VarsSingleMerged/trkWidth200.pdf}
\caption{Distribution of track-jet width in jets for single and merged $b$-jets between 80~GeV to 110~GeV (left) and 200~GeV to 270~GeV (right).}
\label{fig:trkwidthsinglemerged}
\end{figure}

\begin{figure}[tp]
\centering
\includegraphics[width=0.49\textwidth]{FIGS/VarsSingleMerged/NtrktrkWidth080.pdf}
\includegraphics[width=0.49\textwidth]{FIGS/VarsSingleMerged/NtrktrkWidth200.pdf}
\caption{Correlation between jet track multiplicity and track-jet width for single and merged $b$-jets between 80~GeV to 110~GeV (left) and 200~GeV to 270~GeV (right).}
\label{fig:ntrktrkwidthsinglemerged}
\end{figure}

\begin{figure}[tp]
\centering
\includegraphics[width=0.49\textwidth]{FIGS/systematics/Widthsingle_060.pdf}
\includegraphics[width=0.49\textwidth]{FIGS/systematics/Widthmerged_060.pdf}
\caption{Distribution of calorimeter jet width (using topological clusters) for single (left) and merged (right) $b$-jets in two bins of Number of Primary Vertices for jets between 60~GeV to 80~GeV.}
\label{fig:calowidthpileup}
\end{figure}


\begin{figure}[tp]
\centering
\includegraphics[width=0.49\textwidth]{FIGS/systematics/trkWidthsingle_060.pdf}
\includegraphics[width=0.49\textwidth]{FIGS/systematics/trkWidthmerged_060.pdf}
\caption{Distribution of track-jet width for single (left) and merged (right) $b$-jets in two bins of Number of Primary Vertices for jets between 60~GeV to 80~GeV.}
\label{fig:trkwidthpileup}
\end{figure}



{ \em III. Maximum $\Delta R$ between track pairs}
\\[3mm]
Figure~\ref{fig:drmaxsinglemerged} shows the distribution of the maximum $\Delta R$ between track pairs in the jets (Max$\{\Delta R(trk,trk)\}$). Merged $b$-jets show significantly higher values for this variable over a broad range of jet $\pt$. The distinct characteristic of this variable is that the separation between single $b$-jets and merged does not depend on jet $\pt$. In spite of its good discrimination power, we have looked for alternatives to Max$\{\Delta R(trk,trk)\}$ as it is not an infrared safe observable and is sensitive to soft tracks originating from pile-up. 
\\[3mm]

\begin{figure}[tp]
\centering
\includegraphics[width=0.49\textwidth]{FIGS/VarsSingleMerged/drmax080.pdf}
\includegraphics[width=0.49\textwidth]{FIGS/VarsSingleMerged/drmax200.pdf}
\caption{Distribution of the maximum $\Delta R$ between pairs of tracks in jets for single and merged $b$-jets between 80~GeV to 110~GeV (left) and 200~GeV to 270~GeV (right).}
\label{fig:drmaxsinglemerged}
\end{figure}


{ \em IV. $\Delta R$ between the axes of two $k_t$ subjets}
\\[3mm]
The distribution of the $\Delta R$ between the axes of the two exclusive $k_t$ subjets in the jet is shown in Fig.~\ref{fig:drktsinglemerged} for single and merged $b$-jets. In order to build this variable the $k_t$ algorithm~\cite{kt1} is applied to all the tracks associated to the jet using a large $k_t$  distance parameter to ensure that all of them get clustered. The clustering is stopped once it reaches exactly two jets. We observe that this variable also provides good separation, with the advantage of infrared safeness and insensitivity to pile-up. % revealing the two-prong substructure of merged $b$-jets.
\\[3mm]

\begin{figure}[tp]
\centering
\includegraphics[width=0.49\textwidth]{FIGS/VarsSingleMerged/DRkt2axes080.pdf}
\includegraphics[width=0.49\textwidth]{FIGS/VarsSingleMerged/DRkt2axes200.pdf}
\caption{Distribution of the $\Delta R$ between the axes of the two $k_t$ subjets in the jet for single and merged $b$-jets between 80~GeV to 110~GeV (left) and 200~GeV to 270~GeV (right).}
\label{fig:drktsinglemerged}
\end{figure}


{ \em V. $N$-subjettiness variables}
\\[3mm]


%%%%%%%%%%%%%%%%%%%%%%%%%%%%%%%%%%%%%%%%%%%%%%%%%%%%%%%%%%%%%%%%%%%%%%%%%%%%%%%%%%%%%%%%%%%%%%%%%%
As mentioned above, the $N$-subjettiness~\cite{nsubjettiness} is a jet shape that describes the energy flow within a jet. It quantifies the degree to which  radiation is aligned along specified subjet axes. This jet shape was adapted from the event shape $N$-jettiness~\cite{njetti}.

Given candidate subjets directions determined by an external algorithm such as the exclusive $k_t$ procedure~\cite{exclusivekt}, the variables is defined as,


\begin{equation} 
\tau^{(\beta)}_N = \frac{1}{\sum_k {\pt}_k\,(R_0)^{\beta}} \sum_k {\pt}_k (\min \{ \Delta R_{j1,k},\,\Delta R_{j2,k},...,\,\Delta R_{jN,k} \})^{\beta}
\label{eqn:nsubjet}
\end{equation} 

The sum runs over the $k$ constituent particles in a given jet where $p_{T,k}$ are their transverse momenta, and $\Delta R_{j1,k}$ is the distance between the candidate subjet $j1$ and a constituent particle $k$.  $R_0$ is the characteristic jet radius used in the original jet clustering algorithm.
The exponential weight, $\beta$, can optionally be applied to the angular distance computed between the subjets and the jet constituents.  

This jet shape was designed to identify boosted $N$-prong hadronic decays. With $\beta=1$, the definition above indicates that jets with $\tau_N\approx 0$ have all their radiation aligned with the candidate subjet directions and therefore have $N$ (or fewer) subjets. Jets with $\tau_N\gg 0$ have a large fraction of their energy distributed away from the candidate subjet direction and therefore have at least $N+1$ subjets.

To separate boosted hadronic objects from the QCD jet background, one could use the complete set of  $\tau_N$ (with different values of $\beta$) in a multivariate analysis. However, \cite{nsubjettiness} showed that a simple cut on the ratio $\tau_N/\tau_{N-1}$ provides excellent discrimination power for $N$-prong hadronic objects. In particular, $\tau_2/\tau_1$ can identify boosted $W/Z$ and Higgs bosons, with the angular weighting exponent $\beta =1$ providing the best discrimination.

Since eq.~\ref{eqn:nsubjet} is linear in each of the constituent particle momenta, this variable is an infrared- and colliner-safe observable.  In subsequent work~\cite{mininsubjettiness}, Thaler and van Tilburg showed that the initial step of choosing candidate subjet axes is in fact unnecessary. In particular, the quantity in equation~\ref{eqn:nsubjet} can be minimised over the candidate subjet directions, further improving boosted object discrimination.

The definition of $N$-subjettiness is not unique, and different choises can be sued to give different weights to the emissions within a jet. There generalizations of $N$-subjettiness are similar to different ``angularities''~\cite{angularities} used in $e^+e^- \rightarrow$hadrons measurements.


%From Nsubjettiness paper
%use $N$-subjettiness to effectively ``count'' the number of subjets in a jet. Compare to previous jet substructure techniques, $N$-subjettiness  has the advantage of finding jets that contain two or more lobes of energy.
%Subjet candidates where determined by using the exclusive $k_t$ algorithm, forcing it to return exactly $N$ jets.

%%%%%%%%%%%%%%%%%%%%%%%%%%%%%%%%%%%%%%%%%%%%%%%%%%%%%%%%%%%%%%%%%%%%%%%%%%%%%%%%%%%%%%%%%%%%%%%%%%

$N$-subjettiness variables, as described in Ref.~\cite{nsubjettiness}, were originally designed to identify boosted objects, like electroweak bosons and top quarks, decaying into collimated shower of hadrons which a standard jet algorithm would reconstruct as single jets. It is defined as:
\begin{equation} 
\tau_N = \frac{1}{\sum_k {\pt}_k\,R_0} \sum_k {\pt}_k \min \{ \Delta R_{S_1,k},\,\Delta R_{S_2,k},...,\,\Delta R_{S_N,k} \}
\end{equation} 
where $R_0$ is the jet radius used in the jet clustering algorithm and the sum runs over the constituents of the jet. To avoid dependence on pile-up we consider the track-based $n$-subjettiness, where the sum 
 is over the tracks in the $b$-tagged jet. $\Delta R_{S_j,k} $ is the distance in the rapidity-azimuth plane between the axis of subjet $j$ and constituent track $k$. This jet shape variable quantifies to what degree a jet can be regarded as composed of $N$ subjets. For instance, a jet with a two pronged structure, with all tracks clustered along two directions, is expected to have a smaller $\tau_2$ value than a jet with tracks uniformly distributed in $\eta-\phi$ space.

Plots of $ \tau_2$ are shown in Fig.~\ref{fig:tau2singlemerged}. In spite of its expected 2-prong substructure, merged $b$-jets have higher values of $ \tau_2$ than single $b$-jets. The explanation of this behavior can be found in Fig.~\ref{fig:tau2trkwidthsinglemerged}, where its correlation with  track-jet width ($\sim \tau_1$) is shown for single and merged $b$-jets. The two variables are highly correlated and for this reason wider jets  have a larger $ \tau_2$. This suggests to switch from an absolute to a width-normalized
$\tau_2$. Fig.~\ref{fig:tauratiosinglemerged} thus shows the distributions of $\tau_2/\tau_1$. This ratio is often used but, although as expected somewhat larger values are obtained for single than for merged $b$-jets, specially at high $\pt$, we decided not to use this variable as it offers only marginal discrimination. 
\\[3mm]

\begin{figure}[tp]
\centering
\includegraphics[width=0.49\textwidth]{FIGS/VarsSingleMerged/Tau2080.pdf}
\includegraphics[width=0.49\textwidth]{FIGS/VarsSingleMerged/Tau2200.pdf}
\caption{Distribution of $\tau_2$ in jets for single and merged $b$-jets between 80~GeV to 110~GeV (left) and 200~GeV to 270~GeV (right).}
\label{fig:tau2singlemerged}
\end{figure}


\begin{figure}[tp]
\centering
\includegraphics[width=0.49\textwidth]{FIGS/VarsSingleMerged/Tau2trkWidth080.pdf}
\includegraphics[width=0.49\textwidth]{FIGS/VarsSingleMerged/Tau2trkWidth200.pdf}
\caption{Correlation between $\tau _2$ and track-jet width for single and merged $b$-jets between 80~GeV to 110~GeV (left) and 200~GeV to 270~GeV (right).}
\label{fig:tau2trkwidthsinglemerged}
\end{figure}

\begin{figure}[tp]
\centering
\includegraphics[width=0.49\textwidth]{FIGS/VarsSingleMerged/TauRatio080.pdf}
\includegraphics[width=0.49\textwidth]{FIGS/VarsSingleMerged/TauRatio200.pdf}
\caption{Distribution of $\tau_2/\tau_1$ in jets for single and merged $b$-jets between 80~GeV to 110~GeV (left) and 200~GeV to 270~GeV (right).}
\label{fig:tauratiosinglemerged}
\end{figure}


%Variables such as the $\DeltaR$ between the two leading constituents of the jet (those with highest transverse momentum) and the jet eccentricity (the ratio of the principal axes of the jet area) did not show good discrimination either and were not considered for the multivariate study.

{ \em VI. Jet Mass}
\\[3mm]

%%%%%%%%%%%%%%%%%%%%%%%%%%%%%%%%%%%%%%%%%%%%%%%%%%%%%%%%%%%%%%%%%%%%%%%%%%%%%%%%%%%%%%%%%%%%%%%%%
The jet mass, like the linear radial moment, also depends on the radiation pattern of the event. It is the most basic observable for disinguishing massive boosted objects from jets originating from quarks or gluons. The latter are expected to be dominated by wide-angle emissions, with increase probability to see high mass jets initiated from gluons as opposed to quarks~\cite{PhysRevD.79.074012}.  
%%%%%%%%%%%%%%%%%%%%%%%%%%%%%%%%%%%%%%%%%%%%%%%%%%%%%%%%%%%%%%%%%%%%%%%%%%%%%%%%%%%%%%%%%%%%%%%%%%



Figure~\ref{fig:masssinglemerged} shows the distribution of the jet mass for single and merged $b$-jets.
\\[3mm]

\begin{figure}[tp]
\centering
\includegraphics[width=0.49\textwidth]{FIGS/VarsSingleMerged/JetMass080.pdf}
\includegraphics[width=0.49\textwidth]{FIGS/VarsSingleMerged/JetMass200.pdf}
\caption{Distribution of jet mass in~GeV for single and merged $b$-jets between 80~GeV to 110~GeV (left) and 200~GeV to 270~GeV (right).}
\label{fig:masssinglemerged}
\end{figure}



{ \em VII. Number of $k_t$ subjets}
\\[3mm]

%%%%%%%%%%%%%%%%%%%%%%%%%%%%%%%%%%%%%%%%%%%%%%%%%%%%%%%%%%%%%%%%%%%%%%%%%%%%%%%%%%%%%%%%%%%%%%
With the development of the $k_t$ algorithm, subjets were first used in the description of the hadronic final state in $e^+e^-$ annihilation, such as the study of the jet multiplicity at different energy scales~\cite{Catani1992445}. By using the sequential recombination algorithms introduced in the previus section, it is straightforward to define a ``subjet algorithm'' in which the structure of the jet's constituents is resolved using either the same jet finder algorithm or a new one with a fixed (smaller) distance parameter.

The subjet multiplicity $-$ the number of subjets within a jet $-$ provides information on the distribution of energy and multiplicity of particles within a jet. For instance, in~\cite{Snihur1999494} the result of meassuring this ``radiation variable'' on quark- and gluon-initiated jets indicates that gluon-initiated jets tend to have on average higher subjet multiplicity. This result is consistent with the QCD prediction that gluons radiate more than quarks. In the case of this and different other analyses % see for instance  http://iopscience.iop.org/1126-6708/1999/09/009/
the $k_t$ algorithm is rerun for subjet finding.

As an alternative to fixed distance parameter subjets, it is also possible to undo the last step in the recombination sequence~\cite{kt2} in order to identify the decay products of an object.  This approach is used in seveal jet grooming procedures\footnote{Jet grooming comprises dedicated techniques to remove uncorrelated radiation within a jet. A review of these procedures can be found in~\cite{Abdesselam:2010pt}. }, see for instance~\cite{pruning}.

%In contrast to the $k_t$ algorithm, it is not useful just to undo the last stage of C/A clustering: The absence of any momentum scale in its distance measure means that the last clustering often involves soft radiation on the edges of the jet and so, is unrelated to the heavy object's decay. 
%The so-called BDRS method of jet grooming
%http://prl.aps.org/abstract/PRL/v100/i24/e242001
% uses the C/A algorithm. Subjets are then defined by de-clustering the C/A algorithm and evaluating the relative mass of the subjets compared to the parent jet. As a final step, the three hardest identified subjets are recombined to define the resulting ``filtered'' jet.

It is also possible to extend the use of individual subjets in conjunction with more traditional jet shape variables. Using these tools, an inclusive jet shape based on the substructure topology of a single jet, ``$N$-subjettiness''~\cite{nsubjettiness} is defined.
%%%%%%%%%%%%%%%%%%%%%%%%%%%%%%%%%%%%%%%%%%%%%%%%%%%%%%%%%%%%%%%%%%%%%%%%%%%%%%%%%%%%%%%%%%%%%%%%%%


Figure~\ref{fig:nsubjetsinglemerged} shows the distribution of the number of sub-track-jets single and merged $b$-jets.
\\[3mm]

\begin{figure}[tp]
\centering
\includegraphics[width=0.49\textwidth]{FIGS/VarsSingleMerged/Nsubjets080.pdf}
\includegraphics[width=0.49\textwidth]{FIGS/VarsSingleMerged/Nsubjets200.pdf}
\caption{Distribution of the number of $k_t$ sub-track-jets for single and merged $b$-jets between 80~GeV to 110~GeV (left) and 200~GeV to 270~GeV (right).}
\label{fig:nsubjetsinglemerged}
\end{figure}


{ \em VIII. $\Delta R$ between leading constituents}
\\[3mm]

Figure~\ref{fig:drtrk12singlemerged} shows the distribution of the number  $\Delta R$ between leading tracks in the jet for single and merged $b$-jets.
\\[3mm]

\begin{figure}[tp]
\centering
\includegraphics[width=0.49\textwidth]{FIGS/VarsSingleMerged/DRtrk12080.pdf}
\includegraphics[width=0.49\textwidth]{FIGS/VarsSingleMerged/DRtrk12200.pdf}
\caption{Distribution of $\Delta R$ between leading tracks for single and merged $b$-jets between 80~GeV to 110~GeV (left) and 200~GeV to 270~GeV (right).}
\label{fig:drtrk12singlemerged}
\end{figure}


{ \em IX. Jet eccentricity}
\\[3mm]

%%%%%%%%%%%%%%%%%%%%%%%%%%%%%%%%%%%%%%%%%%%%%%%%%%%%%%%%%%%%%%%%%%%%%%%%%%%%%%%%%%%%%%%%%%%%%%%%%%
In defining a jet moment there are several ways to weight the momentum and define the center of the jet. We have defined the jet width as the first moment of the transverse energy with respect to the jet axis; another example of useful combination is the jet pull~\cite{PhysRevLett.105.022001}. But it is also natural to look at higher moments, such as those contained in the covariance tensor,

\[ C = \sum_{i\in jet}\frac{p^i_T|r_i|}{p^{jet}_T} \left( \begin{array}{cc}
 \Delta y^2_i & \Delta y_i \Delta\phi_i \\ 
 \Delta\phi_i \Deltay_i & \Delta \phi^2_i \end{array} \right). \]


Here, $\vect{r}_i = (\Delta y_i, \Delta \phi_i) = \vect{c}_i - \vect{J}$, where $\vect{J} = (y_J,\phi_J)$ is the location of the jet and $\vect{c}_i$ is the position of a cell or particle with transverse momentum $p^i_T$. The eigenvalues $a \geq b$ of this tensor are similar to the semimajor and semiminor axes of an elliptical jet. The jet eccentricity, defined below, is a combination of these eigenvalues, and it is a measure of how elongated is the area of a jet.

\begin{equation} 
e = \sqrt{\frac{(a^2 - b^2)}{a}}
\label{eqn:ecc}
\end{equation}

%No significan difference in eccentricity was found between quark and gluon jets.
%%%%%%%%%%%%%%%%%%%%%%%%%%%%%%%%%%%%%%%%%%%%%%%%%%%%%%%%%%%%%%%%%%%%%%%%%%%%%%%%%%%%%%%%%%%%%%%%%%

Figure~\ref{fig:jeteccsinglemerged} shows the distribution of the jet track-eccentricity for single and merged $b$-jets.
\\[3mm]

\begin{figure}[tp]
\centering
\includegraphics[width=0.49\textwidth]{FIGS/VarsSingleMerged/JetEcc080.pdf}
\includegraphics[width=0.49\textwidth]{FIGS/VarsSingleMerged/JetEcc200.pdf}
\caption{Distribution of the jet eccentricity for single and merged $b$-jets between 80~GeV to 110~GeV (left) and 200~GeV to 270~GeV (right).}
\label{fig:jeteccsinglemerged}
\end{figure}


We also explored the potential improvement of constructing kinematic variables with only displaced tracks, as these are the ones expected to arise from the decay of B-hadrons. Cuts of 2, 2.5 and 3 on the track transverse impact parameter significance were investigated leading however to no gain in discrimation power.

 In Figures~\ref{fig:displacedntrk} and~\ref{fig:displacedtrkwidth} two examples are shown.

\begin{figure}[tp]
\centering
\includegraphics[width=0.49\textwidth]{FIGS/TEMPFigs/DisplacedTracks/ntrk_singlemerged_AllandDisplaced_80-120.pdf}
\includegraphics[width=0.49\textwidth]{FIGS/TEMPFigs/DisplacedTracks/ntrk_singlemerged_AllandDisplaced_260-310.pdf}
\caption{Distribution of the jet track multiplicity single and merged $b$-jets between 80~GeV to 110~GeV (left) and 200~GeV to 270~GeV (right), for all and displaced tracks only.}
\label{fig:displacedntrk}
\end{figure}


\begin{figure}[tp]
\centering
\includegraphics[width=0.49\textwidth]{FIGS/TEMPFigs/DisplacedTracks/trkWidth_singlemerged_AllandDisplaced_80-120.pdf}
\includegraphics[width=0.49\textwidth]{FIGS/TEMPFigs/DisplacedTracks/trkWidth_singlemerged_AllandDisplaced_260-310.pdf}
\caption{Distribution of the track-jet width for single and merged $b$-jets between 80~GeV to 110~GeV (left) and 200~GeV to 270~GeV (right), for all and displaced tracks only.}
\label{fig:displacedtrkwidth}
\end{figure}


\subsection{Further studies using ``ghost-association'' and bigger cone jets}

In order to better understand the behavior observed for $\tau_2$, $\Delta R$ between the axes of $k_T$ subjets and jet eccentricity in anti-$k_T$ 0.4 jets, these variables were studied for other two different scenarios,

\begin{itemize}\addtolength{\itemsep}{-0.4\baselineskip}
\item 
using the active area of jets (with clusters used as input to jet reconstruction).
\item
using bigger 0.6 anti-$k_T$ jets
\end{itemize}
%
in order to enhance the efficiency to capture the decay products in gluon to $b \bar{b}$-jets.

Figures~\ref{fig:tau2GhostAndAntikt6} to~\ref{fig:jeteccGhostAndAntikt6} show distributions of variables mentioned above for single and merged $b$-jets  between 80~GeV to 110~GeV.

\begin{figure}[tp]
\centering
\includegraphics[width=0.49\textwidth]{FIGS/TEMPFigs/Antikt6VarsSingleMerged/Tau2080.pdf}
\includegraphics[width=0.49\textwidth]{FIGS/TEMPFigs/GhostMatchingVarsClus/Tau2080.pdf}
\caption{Distribution of $\tau_2$ for single and merged $b$-jets between 80~GeV to 110~GeV in anti-$k_T$ 0.6 jets using track constituents (left) and anti-$k_T$ 0.4 jets using the active area of the jet, with calorimeter topoclusters as input.}
\label{fig:tau2GhostAndAntikt6}
\end{figure}

\begin{figure}[tp]
\centering
\includegraphics[width=0.49\textwidth]{FIGS/TEMPFigs/Antikt6VarsSingleMerged/DRkt2axes080.pdf}
\includegraphics[width=0.49\textwidth]{FIGS/TEMPFigs/GhostMatchingVarsClus/DRkt2axes080.pdf}
\caption{Distribution of $\Delta R$ between $k_T$ subjets for single and merged $b$-jets between 80~GeV to 110~GeV in anti-$k_T$ 0.6 jets using track constituents (left) and anti-$k_T$ 0.4 jets using the active area of the jet, with calorimeter topoclusters as input.}
\label{fig:drktaxisGhostAndAntikt6}
\end{figure}

\begin{figure}[tp]
\centering
\includegraphics[width=0.49\textwidth]{FIGS/TEMPFigs/Antikt6VarsSingleMerged/JetTrackEccentricity080.pdf}
\includegraphics[width=0.49\textwidth]{FIGS/TEMPFigs/GhostMatchingVarsClus/JetCaloEccentricity080.pdf}
\caption{Distribution of the jet eccentricity for single and merged $b$-jets between 80~GeV to 110~GeV in anti-$k_T$ 0.6 jets using track constituents (left) and anti-$k_T$ 0.4 jets using the active area of the jet, with calorimeter topoclusters as input.}
\label{fig:jeteccGhostAndAntikt6}
\end{figure}




%------------------------------------------------------------------------
\section{Validation of the jet variables in data}\label{sec:gbbValidation}
%------------------------------------------------------------------------

 In order to study the extent to which the simulation reproduces the distributions observed in data for the different variables explored a set of comparison plots is presented. Fig.~\ref{fig:datamcinputvars} shows the distributions of jet track multiplicity, track-jet width and $\Delta R$ between the axes of the two $k_t$ subjets, in two different jet $\pt$ bins in dijet Monte Carlo and data events collected by ATLAS %until summer 2011 ($\mu \approx 6$) overlaid.
during 2011. The distributions are normalized to unit area to allow for shape comparisons. There is a good agreement between data and simulation. It should be remarked that the observed agreement is actually not a direct validation of the description in the MC of the relevant variables, but its convolution with the simulated relative fractions of light-, $c$-, $b$- and $bb$-jets in the $b$-tagged generated jet sample. To some extent, some level of compensation can take place between these two effects.

% For the sake of evaluating the effect of the increasing amount of pile-up in the last period of data-taking, %affected the agreement between experimental data and Monte Carlo
%the same distributions were built using data events recorded in the second half of the year ($\mu \approx 12$).  As expected, no significant variation was observed. In Fig.~\ref{fig:datamcinputvarsItoM} the jet track multiplicity is shown for this period, in the same two $\pt$ bins for comparison.  

\begin{figure}[tp]
\centering
\includegraphics[width=0.49\textwidth]{FIGS/dataMC/FullDataVarNtrkPT080.pdf}
\includegraphics[width=0.49\textwidth]{FIGS/dataMC/FullDataVarNtrkPT200.pdf}
\includegraphics[width=0.49\textwidth]{FIGS/dataMC/FullDataVarTrkWidthPT080.pdf}
\includegraphics[width=0.49\textwidth]{FIGS/dataMC/FullDataVarTrkWidthPT200.pdf}  
\includegraphics[width=0.49\textwidth]{FIGS/dataMC/FullDataVarDRktaxisPT080.pdf}
\includegraphics[width=0.49\textwidth]{FIGS/dataMC/FullDataVarDRktaxisPT200.pdf}  
%\caption{ Distribution of three tracking variables in 2 different jet $\pt$ bins, for experimental data  collected by ATLAS until summer 2011, with $\mu \approx 6$ (solid black points), and simulated data (filled histograms). The ratio data over simulation is shown at the bottom of each plot.}
\caption{ Distribution of three tracking variables in 2 different jet $\pt$ bins, for experimental data  collected by ATLAS during 2011 (solid black points), and simulated data (filled histograms). The ratio data over simulation is shown at the bottom of each plot.}
\label{fig:datamcinputvars}
\end{figure}


%\begin{figure}[tp]
%\centering
%\includegraphics[width=0.49\textwidth]{FIGS/dataMCItoM/VarNtrkPT080.pdf}
%\includegraphics[width=0.49\textwidth]{FIGS/dataMCItoM/VarNtrkPT200.pdf}
%\caption{ Distribution of the jet track-multiplicity in 2 different jet $\pt$ bins, for experimental data from the last period of data-taking, with $\mu \approx 12$ (solid black points), and simulated data (filled histograms). The ratio data over simulation is shown at the bottom of each plot.}
%\label{fig:datamcinputvarsItoM}
%\end{figure}

