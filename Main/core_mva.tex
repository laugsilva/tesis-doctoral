%
%%%%%%%%%%%%%%%%%%%%%%%%%%%%%%%%%%%%%%%%%%%%%%%%%%%%%%%%%%%%%%%%%%%%%%%%%%%%%%%
\chapter{Multivariate Analysis}\label{ch:mva}
%%%%%%%%%%%%%%%%%%%%%%%%%%%%%%%%%%%%%%%%%%%%%%%%%%%%%%%%%%%%%%%%%%%%%%%%%%%%%%%
%


%------------------------------------------------------------------------
\section{The multivariate classifiers}
%------------------------------------------------------------------------

The following multivariate methods were explored:

\begin{itemize}\addtolength{\itemsep}{-0.4\baselineskip}
\item
Likelihood ratio estimators 
\item
Neural Networks (NN)
\item
Boosted decision Trees (BDTs)
\end{itemize}

And different trainings were tested:

\begin{itemize}\addtolength{\itemsep}{-0.4\baselineskip}
\item
Inclusive, with $\pt$-weighting
\item
In bins of jet $\pt$
\end{itemize}


Signal and background jets were not weighted by the dijet samples cross-sections to allow the contribution of subleading lower $\pt$ jets from high $\pt$ events, and thus increase the statistics of merged jets in the low $\pt$ bins. 


Figure~\ref{fig:diffmethodsInclusive} and~\ref{fig:diffmethodsBins} show distributions of the MVA outputs in different bins of jet $\pt$ for the two proposed trainings. In figures~\ref{fig:diffmethodsPerfInclusive} and~\ref{fig:diffmethodsPerfBins} a comparison of the performance of all methods, for inclusive and ``in-bins'', training is illustrated.


\begin{figure}[tp]
\centering
\includegraphics[width=0.32\textwidth]{FIGS/TEMPFigs/MVA_differentMethods/inclusive/NNoutput040_LihoodKDE.pdf}
\includegraphics[width=0.32\textwidth]{FIGS/TEMPFigs/MVA_differentMethods/inclusive/NNoutput080_LihoodKDE.pdf}
\includegraphics[width=0.32\textwidth]{FIGS/TEMPFigs/MVA_differentMethods/inclusive/NNoutput200_LihoodKDE.pdf}
\includegraphics[width=0.32\textwidth]{FIGS/TEMPFigs/MVA_differentMethods/inclusive/NNoutput040_MLP.pdf}
\includegraphics[width=0.32\textwidth]{FIGS/TEMPFigs/MVA_differentMethods/inclusive/NNoutput080_MLP.pdf}
\includegraphics[width=0.32\textwidth]{FIGS/TEMPFigs/MVA_differentMethods/inclusive/NNoutput200_MLP.pdf}
\includegraphics[width=0.32\textwidth]{FIGS/TEMPFigs/MVA_differentMethods/inclusive/NNoutput040_BDT.pdf}
\includegraphics[width=0.32\textwidth]{FIGS/TEMPFigs/MVA_differentMethods/inclusive/NNoutput080_BDT.pdf}
\includegraphics[width=0.32\textwidth]{FIGS/TEMPFigs/MVA_differentMethods/inclusive/NNoutput200_BDT.pdf}
\caption{Distribution of the MVA discriminant outputs, for inclusive training, in single and merged $b$-jets, for low, medium and high jet $\pt$.}
\label{fig:diffmethodsInclusive}
\end{figure}

\begin{figure}[tp]
\centering
\includegraphics[width=0.32\textwidth]{FIGS/TEMPFigs/MVA_differentMethods/bins/NNoutput040_LihoodKDE.pdf}
\includegraphics[width=0.32\textwidth]{FIGS/TEMPFigs/MVA_differentMethods/bins/NNoutput080_LihoodKDE.pdf}
\includegraphics[width=0.32\textwidth]{FIGS/TEMPFigs/MVA_differentMethods/bins/NNoutput200_LihoodKDE.pdf}
\includegraphics[width=0.32\textwidth]{FIGS/TEMPFigs/MVA_differentMethods/bins/NNoutput040_MLP.pdf}
\includegraphics[width=0.32\textwidth]{FIGS/TEMPFigs/MVA_differentMethods/bins/NNoutput080_MLP.pdf}
\includegraphics[width=0.32\textwidth]{FIGS/TEMPFigs/MVA_differentMethods/bins/NNoutput200_MLP.pdf}
\includegraphics[width=0.32\textwidth]{FIGS/TEMPFigs/MVA_differentMethods/bins/NNoutput040_BDT.pdf}
\includegraphics[width=0.32\textwidth]{FIGS/TEMPFigs/MVA_differentMethods/bins/NNoutput080_BDT.pdf}
\includegraphics[width=0.32\textwidth]{FIGS/TEMPFigs/MVA_differentMethods/bins/NNoutput200_BDT.pdf}
\caption{Distribution of the MVA discriminant outputs, for training in bins of jet $\pt$, in single and merged $b$-jets, for low, medium and high jet $\pt$.}
\label{fig:diffmethodsBins}
\end{figure}

\begin{figure}[tp]
\centering
\includegraphics[width=0.49\textwidth]{FIGS/TEMPFigs/MVA_differentMethods/inclusive/MVAs_RejvsEff40.pdf}
\includegraphics[width=0.49\textwidth]{FIGS/TEMPFigs/MVA_differentMethods/inclusive/MVAs_RejvsEff200.pdf}
\caption{Distribution of the MVA discriminant performance for inclusive training, in single and merged $b$-jets, for low and high jet $\pt$.}
\label{fig:diffmethodsPerfInclusive}
\end{figure}

\begin{figure}[tp]
\centering
\includegraphics[width=0.49\textwidth]{FIGS/TEMPFigs/MVA_differentMethods/bins/MVAs_RejvsEff40.pdf}
\includegraphics[width=0.49\textwidth]{FIGS/TEMPFigs/MVA_differentMethods/bins/MVAs_RejvsEff200.pdf}
\caption{Distribution of the MVA discriminant performance for training in bins of jet $\pt$, in single and merged $b$-jets, for low and high jet $\pt$.}
\label{fig:diffmethodsPerfBins}
\end{figure}

%------------------------------------------------------------------------
\section{The input variables}
%------------------------------------------------------------------------

Different groups of input variables were tested. Figure~\ref{fig:difftrainings} shows the performance for three sets of variables for MVA classifier.

\begin{figure}[tp]
\centering
\includegraphics[width=0.32\textwidth]{FIGS/TEMPFigs/MVA_differentTrainings/DiffTrainings_RejvsEff40.pdf}
\includegraphics[width=0.32\textwidth]{FIGS/TEMPFigs/MVA_differentTrainings/DiffTrainings_RejvsEff80.pdf}
\includegraphics[width=0.32\textwidth]{FIGS/TEMPFigs/MVA_differentTrainings/DiffTrainings_RejvsEff200.pdf}
\caption{Distribution of the MVA discriminant performance for three sets of input variables, in single and merged $b$-jets, for low, medium and high jet $\pt$.}
\label{fig:outputinbin}
\end{figure}

%------------------------------------------------------------------------
\section{$\bm{ g\rightarrow b\bar{b}}$ likelihood training and performance}
%------------------------------------------------------------------------

A discriminant between single $b$-jets and merged $b$-jets was built by training a simple likelihood estimator in the context of the Toolkit for Multivariate Data Analysis, TMVA~\cite{Hocker:2007ht}.

A sub-set of the dijet Monte Carlo sample was used for training. After the event and jet selections were performed, the $b$-tagged jets with $|\eta| < 2.1$ were classified as signal (single $b$-jets) or background (merged $b$). % depending on whether one or two $B$-hadrons were found within a 0.4 cone around the jet axis. %For the evaluation of the method the same procedure was followed.
The likelihood training was done in bins of calorimeter jet $\pt$. % and only isolated jets were considered. 
Signal and background jets were not weighted by the dijet samples cross-sections to allow the contribution of subleading lower $\pt$ jets from high $\pt$ events, and thus increase the statistics of merged jets in the low $\pt$ bins. For the evaluation of the method the same procedure was followed.

Several combinations of the tracking and jet shape variables studied in the previous section were tested as input variables. We found that the following three offer the best performance:
% that the jet track multiplicity, the track-jet width and $\Delta R$  between the axes of the two $k_t$ subjets offered the best performance. 
\begin{enumerate}\addtolength{\itemsep}{-0.4\baselineskip}
\item
Jet track multiplicity
\item
Track-jet width
\item
$\Delta R$ between the axes of 2 $k_t$ subjets within the jet
\end{enumerate}
%
A requirement of at least two matching tracks was imposed to all $b$-tagged jets in order to build the third variable listed. This cut was applied in both training and testing samples.

The distribution of the likelihood output for single and merged $b$-jets is shown in  Fig.~\ref{fig:outputinbins} for low, medium and high transverse momentum jets.

\begin{figure}[tp]
\centering
\includegraphics[width=0.32\textwidth,viewport=40 0 540 550]{FIGS/Likelihood/NNoutput040_LihoodKDE.pdf}
%\includegraphics[width=0.32\textwidth]{FIGS/Likelihood/NNoutput040_LihoodKDE.pdf}
\includegraphics[width=0.32\textwidth,viewport=40 0 540 550,clip]{FIGS/Likelihood/NNoutput080_LihoodKDE.pdf}
\includegraphics[width=0.32\textwidth,viewport=40 0 540 550,clip]{FIGS/Likelihood/NNoutput200_LihoodKDE.pdf}  
\caption{Distribution of the $g\rightarrow b \bar{b}$ likelihood output for single and merged $b$-jets for low, medium and high $\pt$ jets.}
\label{fig:outputinbins}
\end{figure}

The performance of the $g\rightarrow b \bar{b}$ tagger in the simulation can be  %seen in Fig.~\ref{fig:performanceinbins}. In this plot the rejection ($1/\epsilon_{bkg}$) of merged $b$-jets from gluon splitting is shown as a function of single $b$-jet efficiency for the eight bins of jet $\pt$ mentioned in section~\ref{sec:EventSelection}. The performance improves with $\pt$:
displayed in a plot of rejection ($1/\epsilon_{bkg}$) of merged $b$-jets as a function of single $b$-jet efficiency, where $\epsilon_{bkg}$ is the probability that a $b \bar{b}$-jet passes the tagger. This is shown in Fig.~\ref{fig:performanceinbins} for the eight bins of jet $\pt$ mentioned in section~\ref{sec:EventSelection}. The performance improves with $\pt$:


\begin{itemize}\addtolength{\itemsep}{-0.4\baselineskip}
\item
$\pt>40$ GeV: %\\[1mm]
rejection above 8 at 50\% eff.
% over 85\% rejection at 50\% eff.
\item
$\pt>60$ GeV: %\\[1mm]
rejection above 10 at 50\% eff.
 %90\% rejection at 50\% eff.
\item
$\pt>200$ GeV: %\\[1mm]
rejection above 30 at 50\% eff.
 %over 95\% rejection at 50\% eff.
\end{itemize}


\begin{figure}[tp]
\centering
\includegraphics[width=0.7\textwidth]{FIGS/Likelihood/KDE_RejvsEff.pdf}
\caption{Rejection of $g\rightarrow b \bar{b}$ merged $b$-jets as a function of $b$-jet efficiency for dijet events in 8 jet $\pt$ bins.}
\label{fig:performanceinbins}
\end{figure}

The likelihood was trained with jets that had been first tagged by the MV1 algorithm. In order to use the  $g\rightarrow b \bar{b}$ classifier for jets tagged by another tagger a new training is required.

The rejection of merged jets attained as a function of $\pt$ for the 50\% and 60\% efficiency working points are summarized in Table~\ref{tb:rejection}, together with their relative statistical error. These are propagated from the Poisson fluctuations of the number of events in the merged and single $b\bar{b}$ distributions. The error is slightly lower for the 60\% efficiency working point because a higher efficiency allows for a greater number of Monte Carlo events to measure the performance. %the higher the efficiency, the larger the number of Monte Carlo events available to measure the performance.



\begin{table}[!hbt] %[h]
\renewcommand{\arraystretch}{1.2}
\centering
\begin{tabular}{ | c || c | c || c | c||}
  \hline
  Jet $\pt$ & \multicolumn{2}{c||}{single $b$-jet efficiency 50\%} & 
            \multicolumn{2}{c||}{single $b$-jet efficiency 60\%}\\ \cline{2-5}
    (GeV )  & Rejection & ~stat.err.~ & Rejection & ~stat.err.~ \\ \hline
%   40 - 60 &  ~7.6 &  4\%  &  ~5.4  &  3\%    \\ 
%   60 - 80 &  10.4 &  4\%  &  ~7.3  &  4\%    \\ 
%   80 - 110&  13.9 &  5\%  &  ~9.4  &  4\%    \\ 
%  110 - 150&  18.7 &  5\%  &  12.0  &  4\%    \\ 
%  150 - 200&  23.2 &  5\%  &  14.0  &  5\%    \\ 
%  200 - 270&  29.5 &  7\%  &  15.6  &  6\%    \\ 
%  270 - 360&  36.4 &  7\%  &  18.7  &  6\%    \\ 
%  360 - 480&  41.4 &  8\%  &  18.2  &  8\%    \\ \hline
   40 - 60 &  ~8 &  4\%  &  ~5  &  3\%    \\ 
   60 - 80 &  ~10 &  4\%  &  ~7  &  4\%    \\ 
   80 - 110&  ~14 &  5\%  &  ~9  &  4\%    \\ 
  110 - 150&  19 &  5\%  &  12  &  4\%    \\ 
  150 - 200&  23 &  5\%  &  14  &  5\%    \\ 
  200 - 270&  30 &  7\%  &  16  &  6\%    \\ 
  270 - 360&  36 &  7\%  &  19  &  6\%    \\ 
  360 - 480&  41 &  8\%  &  18  &  8\%    \\ \hline
\end{tabular}
\caption{The merged $b$-jet rejection for the 50\% and 60\% efficiency working points in bins of $\pt$.}
\label{tb:rejection}
\end{table}







%------------------------------------------------------------------------
\section{Systematic uncertainties}\label{sec:gbbSystematics}
%------------------------------------------------------------------------

The development, training and performance determination of the tagger is based on simulated events. Although the agreement between simulation and data explored in section~\ref{sec:validation} is a necessary validation condition, it is also important to investigate how the tagger performance depends on systematics relevant in the data. In particular we have considered:
%
%There are uncertainties in the data that, although not spoiling the data/MC agreement, might change the performance of the tagger. This uncertainties would also be present in an eventual future data-driven determination of the performance as  the data are not known better that this.The following systematic effects have been studied and their influence on the tagger performance ascertained:
%
%In order to study the systematic uncertainties in the method the following contributions were evaluated:
\begin{itemize}\addtolength{\itemsep}{-0.4\baselineskip}
\item
presence of additional interactions (pile-up)
\item
uncertainty in the $b$-jet tagging efficiency % and b-jet energy scale?
\item
uncertainty in the track reconstruction efficiency
\item
uncertainty in the track transverse momentum resolution
\item
uncertainty in the jet transverse momentum resolution  
%\item
%the effect of jet isolation
%\item
%other $\Delta R$ cuts for B-labeling and matching
\end{itemize}

{ \em I. Pile-up}
\\[3mm]
  The size of this effect was studied by comparing the performance of the likelihood discriminant with $b$-jets in events with small (1-9) and large (9-20) number of primary vertices. 
The comparison of the performance in these two sub-samples can be seen in Fig.~\ref{fig:performanceinbinsMu}. 
As expected from the use of tracking (as opposed to calorimeter) variables no significant dependence with pile-up is observed within statistics. Of the 16 determinations (2 working points with 8 $\pt$ bins each) of performance differences between high and low number of primary vertices events, it is observed that 6 of them are positive and 10 negative, with a global mean of 0.3\%. We conclude that the effect is negligible compared to other source of uncertainties.
%

\begin{figure}[tp]
\centering
\includegraphics[width=0.49\textwidth]{FIGS/systematics/50BinsLlhoodKDE_ISO_PileUp_rejvseff040.pdf}
\includegraphics[width=0.49\textwidth]{FIGS/systematics/50BinsLlhoodKDE_ISO_PileUp_rejvseff060.pdf}
\caption{Rejection of $g\rightarrow b \bar{b}$ merged b-jets as a function of $b$-jet efficiency in bins of $N_{\rm vtx}$ for two low jet $\pt$ bins.}
\label{fig:performanceinbinsMu}
\end{figure}

\vspace{3mm}
{\em II. b-tagging efficiency} %energy scale and 
\\[3mm]
The performance of heavy-flavor tagging in Monte Carlo events is calibrated to experimental data by means of the scale factors (SFs) measured by the $b$-tagging group. %scale factors defined as the ratio of the heavy-flavor tagging efficiency in data over that in Monte Carlo (MC) for the different jet flavors, and probably also as a function of jet pT and η 
Such a measurement carries a systematic uncertainty, and in order to estimate its effect a conservative approach is followed: %https://twiki.cern.ch/twiki/bin/viewauth/AtlasProtected/BTaggingCalibrationDataInterface
the SFs are varied in all the $\pt$ bins simultaneously by one standard deviation both in the up and down directions. The result of this procedure for the distribution of two of the tracking variables used in our discriminant is illustrated in Fig.~\ref{fig:btaggingSFs}. 

The effect of the $b$-tagging calibration uncertainty on the likelihood peformance is $< 1$\%, negligible with respect to the statistical uncertainty as it can be seen in Fig.~\ref{fig:btaggingSFsPerf}.
This was indeed expected. The scale factors depend on the true flavor of the jet and on its $\pt$, but these are basically constant in the performance determination, which is based on single flavor (true $b$-) jets classified in $\pt$-bins.

%\textwidth,viewport=45 45 696 672,clip
\begin{figure}[tp]
\centering
\includegraphics[width=0.49\textwidth]{FIGS/systematics/BTagCalib_DataVarNtrkPT080.pdf}
%\includegraphics[width=0.49\textwidth]{FIGS/systematics/BTagCalib_DataVarNtrkPT200.pdf}
%\includegraphics[width=0.49\textwidth]{FIGS/systematics/BTagCalib_DataVarTrkWidthPT080.pdf}
%\includegraphics[width=0.49\textwidth]{FIGS/systematics/BTagCalib_DataVarTrkWidthPT200.pdf}
\includegraphics[width=0.49\textwidth]{FIGS/systematics/BTagCalib_DataVarDRktaxisPT080.pdf}
%\includegraphics[width=0.49\textwidth]{FIGS/systematics/BTagCalib_DataVarDRktaxisPT200.pdf}
\caption{The effect of a variation in the $b$-tagging Scale Factors on the tracking variables distributions. Scale Factors were varied up (down) by 1-sigma to evaluate the systematic uncertainty from this source. The ratio data over MC is shown for MC {\sc pythia} with SFs varied up (circles) and down (triangles).}
\label{fig:btaggingSFs}
\end{figure}

\begin{figure}[tp]
\centering
\includegraphics[width=0.49\textwidth]{FIGS/systematics/LlhoodKDE_ISO_BTagCalibTest_rejvseff060.pdf}
\includegraphics[width=0.49\textwidth]{FIGS/systematics/LlhoodKDE_ISO_BTagCalibTest_rejvseff270.pdf}
\caption{Rejection of $g\rightarrow b \bar{b}$ merged b-jets as a function of $b$-jet efficiency with and without scale factors.}
\label{fig:btaggingSFsPerf}
\end{figure}

%

\vspace{3mm}
{ \em III. Track reconstruction efficiency}
\\[3mm]
This uncertainty arises from the limit in the understanding of the material layout of the Inner Detector. To test its impact a fraction of tracks determined from the track efficiency uncertainty was randomly removed following the method in Ref.~\cite{JetMassNote}.%https://cdsweb.cern.ch/record/1344082?ln=en

The tracking efficiency systematics are given in bins of track $\eta$. For tracks with $p_{\rm{T}}^{\rm{track}} > 500$~MeV the uncertainties are independent of $\pt$:  2\% for $|\eta^{\rm{track}}|<1.3$, 3\% for $1.3<|\eta^{\rm{track}}|<1.9$, 4\% for $1.9<|\eta^{\rm{track}}|<2.1$, 4\% for $2.1<|\eta^{\rm{track}}|<2.3$ and 7\% for $2.3<|\eta^{\rm{track}}|<2.5$~\cite{chargemultiplicity}. All numbers are relative to the corresponding tracking efficiencies.  
%https://cdsweb.cern.ch/record/1286839
%http://arxiv.org/abs/1012.5104

The tracking variables were re-calculated and the performance of the nominal likelihood was evaluated in the new sample with worse tracking efficiency. The rejection-efficiency plots, shown in Fig.~\ref{fig:trackefficiency}, show a small degradation of the performance which is comparable to the statistical uncertainty. The effect is however systematically present over all 16 $\pt$ bin/working points, without a clear $\pt$ dependence. We have thus taken the average over $\pt$, and obtained a global systematic uncertainty of 4\% both for the 50\% and 60\% efficiency working points.

\begin{figure}[tp]
\centering
\includegraphics[width=0.49\textwidth]{FIGS/systematics/LlhoodKDE_ISO_TrackingUncertaintyTest_rejvseff080.pdf}
\includegraphics[width=0.49\textwidth]{FIGS/systematics/LlhoodKDE_ISO_TrackingUncertaintyTest_rejvseff200.pdf}
\caption{Rejection of $g\rightarrow b \bar{b}$ merged $b$-jets as a function of $b$-jet efficiency showing shift in likelihood performance caused by a reduction in the tracking efficiency .}
\label{fig:trackefficiency}
\end{figure}

\vspace{3mm}
{\em IV. Track momentum resolution}
\\[3mm]
The knowledge of the track momentum resolution is limited by the precision both in the material description of the Inner Detector and in the mapping of the magnetic field. Its uncertainty propagates to the kinematic variables used in the 
$g\rightarrow b \bar{b}$ tagger. In order to study this effect, track momenta are over-smeared according to the measured resolution uncertainties before computing the rejection. The actual smearing is done in 1/$\pt$, with an upper bound to the resolution uncertainty given by $\sigma$(1/$\pt$)=0.02/$\pt$~\cite{ATLAS-CONF-2010-009}. The effect is found to be negligible. %with respect to the statistical uncertainty.

\vspace{3mm}
{ \em V. Jet transverse momentum resolution}
\\[3mm]
The jet momentum resolution was measured for 2011 data and found to be in agreement with the predictions from the {\sc pythia8}-based simulation~\cite{JER2011}. The precision of this measurement, determined in $\pt$ and $\eta$ bins,is typically 10\%.
The systematic uncertainty due to the calorimeter jet $\pt$ resolution was estimated by over-smearing the jet $4$-momentum in the simulated data, without changing jet $\eta$ or $\phi$ angles. The performance is found to globally decrease by 6\%, without a particular $\pt$ dependence.

%\begin{figure}[tp]
%\centering
%\includegraphics[width=0.49\textwidth]{FIGS/systematics/LlhoodKDE_ISO_SmearedJetPtTest_rejvseff060.pdf}
%\includegraphics[width=0.49\textwidth]{FIGS/systematics/LlhoodKDE_ISO_SmearedJetPtTest_rejvseff270.pdf}
%\caption{Rejection of $g\rightarrow b \bar{b}$ merged b-jets as a function of $b$-jet efficiency for jets with smeared $\pt$.}
%\label{fig:jetresolution}
%\end{figure}
\vspace{3mm}
The different contributions to the systematic uncertainty on the $g\rightarrow b \bar{b}$ rejection are summarized in Table~\ref{tb:systematics}.
\begin{table}[!hbt] %[h]
\renewcommand{\arraystretch}{1.2}
\centering
\begin{tabular}{ | c | c |}
\hline
  ~~~~~~~Systematic source~~~~~~~ &~~Uncertainty~~\\ \hline
  pile-up          &  neglible     \\ 
  $b$-tagging efficiency     &  neglible     \\ 
  track reconstruction efficiency  &    4\%        \\ 
  track $\pt$ resolution &  neglible     \\
  jet $\pt$ resolution  &    6\%        \\ \hline 
\end{tabular}
\caption{Systematic uncertainities in the merged $b$-jet rejection (common to both the 50\% and the 60\% efficiency working points).}
\label{tb:systematics}
\end{table}


%------------------------------------------------------------------------
\section{Isolation studies}
%------------------------------------------------------------------------
Although the tagger was derived with isolated jets it can also be applied to non-isolated jets. Studies were performed to evaluate the likelihood rejection in $b$-jets with close-by jet with $\pt$ between 7 GeV at electromagnetic scale scale and 90$\%$ of the $b$-jet $\pt$. The results can be seen in  Fig.~\ref{fig:testisolation}. The presence of close-by jets with a susbtancial fraction of the $b$-jet pt worsens the performance in more than 50$\%$ at very high $\pt$. 

\begin{figure}[tp]
\centering
\includegraphics[width=0.4\textwidth]{FIGS/systematics/DiffIsolationCutsKDE_RejvsEff40.png}
\includegraphics[width=0.4\textwidth]{FIGS/systematics/DiffIsolationCutsKDE_RejvsEff270.png}
\caption{Rejection of $g\rightarrow b \bar{b}$ merged b-jets as a function of $b$-jet efficiency for two different isolation cuts.}
\label{fig:testisolation}
\end{figure}


%------------------------------------------------------------------------
\section{Other Monte Carlo generators}\label{sec:otherMC}
%------------------------------------------------------------------------
%{\em Uncertainties due to the event modelling in the Monte Carlo generators}

The development, training and performance determination of the tagger has been done using Monte Carlo events generated with the {\sc pythia8} event simulator, interfaced to the {\sc geant4} based simulation of the ATLAS detector. An immediate question is what the performance would be if studied with a different simulation. In this section we investigate this question for the Perugia tune of {\sc pythia8} and the {\sc herwig++} event generators.

Fig.~\ref{fig:performanceotherMC} shows a comparison of the likelihood rejection, at the 50\% efficiency working point,  between nominal {\sc pythia} and the alternative simulations % for four selected $\pt$ bins covering the full kinematic range. 
as a function of the jet $\pt$ . The larger errors are due to the reduced statistics available, which are even lower for the Perugia case than for {\sc herwig}.



%In order to account for the dependence on different generator models and tunes the likelihood performance was tested using other Monte Carlo simulations. Results with nominal Pythia were compared to the performance derived with dijet samples generated with Herwig++ and with the Perugia tune of Pythia. The comparison can be seen in Fig.~\ref{fig:performanceotherMC}. 

The performance in {\sc herwig} shows a systematic trend, with agreement at low $\pt$ and increasingly poor performances compared to {\sc pythia} as $\pt$ grows. For the Perugia tune, on the other hand, there is no definite behavior, with the performance fluctuating above or below the nominal simulation for different $\pt$ bins consistently with the statistical uncertainties.

The reason for the systematic difference observed between the performances of {\sc pythia} and {\sc herwig} can be traced to the extent with which jets are accurately modelled. Fig.~\ref{fig:herwigdatamc} compares the measured jet track multiplicity distributions in $b$-tagged jets and the prediction from both simulations, for low and high $\pt$ jets. It is observed that indeed {\sc herwig++} does not correctly reproduce the data, particularly at high $\pt$. The level of agreement is found to be better for track-jet width and the $\Delta R$ between the axes of the two $k_t$ subjets in the jet, the two other variables used for discrimination.



%the comparion between jet distribution in experimental data and  Herwig++ Monte Carlo events was performed for $b$-tagged jets. The results for the jet track-multiplicity can be seen in Fig.~\ref{fig:herwigdatamc}. We find that Herwig++ does not reproduce data jet distributions correctly at medium and high \pt.



\begin{figure}[tp]
\centering
\includegraphics[width=0.49\textwidth]{gbbRejection_vs_PT_3MonteCarlos_50Eff.pdf}
%\includegraphics[width=0.49\textwidth]{FIGS/systematics/newInterpLlhoodKDE_ISO_DiffMCGen_rejvseff040.pdf}
%\includegraphics[width=0.49\textwidth]{FIGS/systematics/newInterpLlhoodKDE_ISO_DiffMCGen_rejvseff060.pdf}
%\includegraphics[width=0.49\textwidth]{FIGS/systematics/newInterpLlhoodKDE_ISO_DiffMCGen_rejvseff200.pdf}
%\includegraphics[width=0.49\textwidth]{FIGS/systematics/newInterpLlhoodKDE_ISO_DiffMCGen_rejvseff360.pdf}
\caption{Rejection of $g\rightarrow b \bar{b}$ merged $b$-jets as a function of jet $\pt$ for different Monte Carlo generators, at the 50\% efficiency working point.}
\label{fig:performanceotherMC}
\end{figure}

\begin{figure}[tp]
\centering
\includegraphics[width=0.49\textwidth]{FIGS/systematics/DataVarNtrkPT040.pdf}
\includegraphics[width=0.49\textwidth]{FIGS/systematics/DataVarNtrkPT200.pdf}
\caption{Distribution of the jet track multiplicity in 2 different jet $\pt$ bins, for experimental data  collected during 2011 (solid black points) and {\sc herwig}++ events (solid violet triangules). The ratio data over {\sc herwig}++ simulation is shown at the bottom of the plot. {\sc pythia} distribution is also shown for reference.}
\label{fig:herwigdatamc}
\end{figure}



%
%%%%%%%%%%%%%%%%%%%%%%%%%%%%%%%%%%%%%%%%%%%%%%%%%%%%%%%%%%%%%%%%%%%%%%%%%%%%%%%
% Fractio of gluon splitting in data
%%%%%%%%%%%%%%%%%%%%%%%%%%%%%%%%%%%%%%%%%%%%%%%%%%%%%%%%%%%%%%%%%%%%%%%%%%%%%%%
%
\chapter{??Fraction of gluon-splitting jets in data??}\label{ch:gbbfraction}

%------------------------------------------------------------------------
\section{??Template fits??}\label{sec:FractionSystematics}
%------------------------------------------------------------------------

%------------------------------------------------------------------------
%\section{Systematic uncertainties}\label{sec:FractionSystematics}
%------------------------------------------------------------------------
