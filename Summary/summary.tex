%
%%%%%%%%%%%%%%%%%%%%%%%%%%%%%%%%%%%%%%%%%%%%%%%%%%%%%%%%%%%%%%%%%%%%%%%%%%%%%%%
% Conclusions
%%%%%%%%%%%%%%%%%%%%%%%%%%%%%%%%%%%%%%%%%%%%%%%%%%%%%%%%%%%%%%%%%%%%%%%%%%%%%%%
%
\chapter{Conclusions}\label{ch:conclusions}

A multivariate discriminant to identify isolated $b$-tagged jets containing two $B$-hadrons is presented. These jets are expected to arise when a gluon splits into a close-by $b\bar{b}$-pair. The method exploits the kinematic differences between ``merged'' $b\bar{b}$-jets and ``single'' $b$-jets, combining track-based jet shape and jet substructure variables in a likelihood classifier.

The tagger training and performance results are based on simulated events. Several variables were investigated and those showing the best discrimination power were selected for the multivariate analysis. The Monte Carlo distributions of the explored variables were validated using experimental data corresponding to an integrated luminosity of 4.7~fb$^{-1}$ recorded by the ATLAS experiment during 2011. The agreement between data and simulation is excellent. 

 The peformance of the tagger in Monte Carlo events was studied in bins of the calorimeter jet \pt, achieving a rejection of merged jets of over 95\% (90\%) for a 50\% single $b$-jet efficiency for jets with $\pt>150$ GeV ($\pt>60$ GeV).


This tool provides a handle to investigate QCD $b\bar{b}$ production and to reduce backgrounds in physics channels involving $b$-quarks in the final state.  Future improvements comprise the study of further discrimant variables, the extension to non-isolated jets using the concept of ghost-particle matching and active area of a jet~\cite{CatchmentArea} for track-to-jet association and labeling, the calibration of the tagger with data, and its application to measure the fraction of gluon-splitting jets in QCD $b$-jet production.







