%
%%%%%%%%%%%%%%%%%%%%%%%%%%%%%%%%%%%%%%%%%%%%%%%%%%%%%%%%%%%%%%%%%%%%%%%%%%%%%%%
% Conclusions
%%%%%%%%%%%%%%%%%%%%%%%%%%%%%%%%%%%%%%%%%%%%%%%%%%%%%%%%%%%%%%%%%%%%%%%%%%%%%%%
%
\chapter{Summary and conclusions}\label{ch:conclusions}

In the course of the present thesis a new method was developed  %allowing the identification of $b$-jets containing two $b$-hadrons was developed.
to identify $b$-jets containing two $b$-hadrons which do not arise from heavy flavour production at the hard interaction but mainly via subsequent $g \rightarrow b\bar{b}$ branchings.

The method exploits the expected kinematic differences between double $b$-hadron (``merged'') jets and single $b$-jets, combining a set of discriminating variables in a multivariate classifier.  The differences between single and merged jets originate in the two-subjet structure of merged jets,  %which, in QCD, are expected to arise mainly from a gluon splitting into a  close-by $b\bar{b}$-pair. 
which tend to have higher multiplicity and larger width. Several jet shape and substructure variables accounting for these envisaged characteristics were investigated in order to obtain the best single-merged discrimination.  Due to the noisy environment of the hadron collisions at the LHC track-based variables were preferred over calorimeter variables.   %A good agreement with a data sample of 4.7~fb$^{-1}$ recorded by the ATLAS during 2011 is observed for all the variables explored.

%The multivariate classifier was trained using simulated QCD events. 
A likelihood ratio estimator was trained using simulated QCD events.   Based on discrimination power, correlation and pile-up dependence three input variables were selected for the tagger training: the jet track multiplicity, the track-jet width and the $\Delta R$ between the axes of two $k_t$ subjets in the jet. 
%The tagger training and performance results are based on simulated events. Several variables were investigated and those showing the best discrimination power were selected for the multivariate analysis. %The Monte Carlo distributions of the explored variables were validated using experimental data corresponding to an integrated luminosity of 4.7~fb$^{-1}$ recorded by the ATLAS experiment during 2011. The agreement between data and simulation is excellent. 
 The peformance of the tagger in Monte Carlo events was studied in bins of the calorimeter jet $\pt$, achieving a rejection of merged jets of over 95\% (90\%) for a 50\% single $b$-jet efficiency for jets with $\pt>150$ GeV ($\pt>60$ GeV).
A comprehensive study of the sources of systematic uncertainties in merged $b$-jet rejection  %were evaluated for the 50\% and 60\% single $b$-jet efficiency working points, 
was performed, the most relevant being the tracking efficiency and the jet energy scale and resolution with %an average performance variation of 4\%, 5\% and 5\%, respectively. 
a contribution to the uncertainty of  4\%, 5\% and 5\%, respectively. 
Other sources such as pile-up or the uncertainties in the track momentum resolution and the $b$-jet tagging efficiency proved to be negligible.


The Monte Carlo distributions of the explored variables were validated using experimental data corresponding to an integrated luminosity of 4.7~fb$^{-1}$ recorded by the ATLAS experiment during 2011. The agreement between data and simulation is excellent.

%\emph{Monte Carlo templates were used to fit the likelihood distribution in data in order to obtain the fraction of merged $b$-jets in the data sample. This measurement was performed by means of unbinned maximum likelihood fits. The systematic uncertainties that most affected the method were the uncertainties in the jet energy scale and the jet energy resolution, with an average variation of 2\% and 1\% respectively.}
The tool developed was used to measure the fraction of merged $b$-jets in QCD $b$-jet production. The results obtained are in very good agreement with the theoretical prediction from a QCD parton shower simulation of $pp$ collisions.

This tool provides a handle to investigate QCD $b\bar{b}$ production and to reduce backgrounds in %physics channels involving $b$-quarks in the final state. 
 Standard Model physics analyses that rely on the presence of single $b$-jets in the final state, such as top quark physics (either in the $t\bar{t}$ or the single top channels) or associated Higgs production ($WH\rightarrow\ell\nu b\bar{b}$ and $ZH\rightarrow\nu\nu b\bar{b}$). %These processes  suffer from backgrounds that can be in part removed by a merged $b$-jet tagger.
 Jets containing a single $b$-quark or antiquark %Single $b$-jets 
also enter in many BSM collider searches, the ability to distinguish single $b$-jets from jets containing two $b$-hadrons is thus here of wide application to reduce SM backgrounds giving rise to close-by $b\bar{b}$ pairs.

In order to expand up the results presented here, and to make further advancements in the implementation of the tagger in physics analyses the following improvements should be made: the extension to non-isolated jets using the concept of ghost-particle matching and active area of a jet for track-to-jet association and labeling and the calibration of the tagger with data. %, and its application to measure the fraction of gluon-splitting jets in QCD $b$-jet production.
Nontheless, the study presented in this thesis demonstrates that jet substructure variables can provide a good handle for gluon splitting identification in physics searches within ATLAS.






