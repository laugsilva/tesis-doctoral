%
%%%%%%%%%%%%%%%%%%%%%%%%%%%%%%%%%%%%%%%%%%%%%%%%%%%%%%%%%%%%%%%%%%%%%%%%%%%%%%%
% Object reconstruction and event selection
%%%%%%%%%%%%%%%%%%%%%%%%%%%%%%%%%%%%%%%%%%%%%%%%%%%%%%%%%%%%%%%%%%%%%%%%%%%%%%%
%

\chapter{Object reconstruction and $b$-Tagging }


Experimental data and simulated events were both reconstructed using the latest version available of the ATLAS software. In this chapter we brielfy describe the reconstruction of the two key objects used in this analysis, namely jets and tracks.
We describe how $b$-Tagging is done in ATLAS.


%------------------------------------------------------------------------
\section{Jets}\label{sec:ObjSelection}
%------------------------------------------------------------------------


Jets are reconstructed using the anti-$k_t$ jet algorithm~\cite{antiktalg} with a distance parameter $R = 0.4$, using calorimeter topological clusters~\cite{topoClusters} as input.  %Topological clusters are built starting from seed calorimeter cells with a signal at least four times higher than the root-mean-square (RMS) of the total noise contribution, consisting of electronic and pile-up noise corresponding to an average %luminosity number of interactions of $\mu=8$. Cells neighbouring the seed which have a signal-to-RMS-noise-ratio of two are then iteratively added. Finally, all nearest neighbour cells are added to the cluster without any threshold. 
Several quality criteria are applied to eliminate ``fake'' jets that are %produced by problematic calorimeter behaviour~\cite{ATLAS-CONF-2010-038}.
caused by noise bursts in the calorimeters and energy depositions belonging to a previous bunch crossing~\cite{ATLAS-CONF-2012-020}.
%~\cite{ATLAS-CONF-2010-038}.
%In particular the jet is rejected if 90\% of the jet energy is distributed over less than 6 calorimeter cells (this handles fake jets caused by sporadic noise bursts in the LAr calorimeter). Jets are also discarded if the fraction of jet energy from LAr calorimeter cells flagged as problematic is greater than 0.8. If the fraction of energy deposited in the electromagnetic calorimeter is greater than 0.95 the jet is eliminated (this removes fake jets caused by noise bursts). 
%In order to reject jets reconstructed from large out-of-time energy deposits a loose cut on the jet time, measured with respect to the event time, is imposed.
%Jets are required to have an energy-squared-weighted cell time to be within two beam bunch crossings, i.e. out-of-time jets are selected with jet time $> 50$~ns.


The jet energies are corrected for inhomogeneities and for
the non-compensating nature of the calorimeter by using \pt- and $\eta$-dependent calibration factors determined from Monte Carlo simulation~\cite{JESnote}. This calibration is referred to as the EM+JES scale.
Using test beam results, in-situ track and calorimeter measurements, estimations of pile-up energy depositions, and detailed Monte Carlo comparisons, an uncertainty on the absolute jet energy scale was estabished. This uncertainty is smaller than $\pm 10\%$ for $\eta < 2.8$ and $\pt > 20$ GeV. More sophisticated techniques undergoing commissioning, such as local cluster weighting, are expected to considerably improve the jet energy uncertainty and resolution~\cite{CSC}. 

%------------------------------------------------------------------------
\section{Tracks}\label{sec:ObjSelection}
%------------------------------------------------------------------------


The tracks of charged particles with a pseudorapidity $|\eta| < 2.5$ are reconstructed in the the Inner Detector. It is composed of a barrel, consisting of 3 Pixel layers, 4 double layers of single-sided silicon strip sensors, and 73 layers of Transition Radiation Tracker straws concentric with the beam, plus a system of disks on each end of the barrel, occupying in total a cylindrical volume around the interaction point of radius of 1.15~m and length of 7.024~m. The Pixel detector's innermost layer is located at a radius of 5 cm from the beam axis, has a position resolution of approximately 10~$\mu$m in the $r-\phi$ plane and 115 $\mu$m along the beam axis ($z$). %Tracks with $p^{\rm{track}}_{\rm{T}} > 150$
Tracks with $p^{\rm{track}}_{\rm{T}} > 400$ MeV and consistent with the beamspot are associated in primary vertices via a finding/fitting sequential algorithm. Several primary vertices can be reconstructed per event due to the presence of in-time pile-up. The one with at least five associated tracks, a $z$ position within 100~mm of the ATLAS geometrical center, and the largest $\sum_{\rm trk}\pt^2$, is selected as the one associated to the hard interaction.
%are selected as good vertices.

%-------------------------------------------------------------------
\section{ $\bm b$-jet Tagging}\label{sec:btagging}
%------------------------------------------------------------------------

Jets are classified as $b$-quark candidates by the ATLAS MV1 $b$-tagging algorithm, based on a neural network that combines the information from three high-performance taggers: IP3D, SV1 and JetFitter \cite{ATLAS-CONF-2011-102}.  These three tagging algorithms use a likelihood ratio technique in which input variables are compared to smoothed normalized distributions for both the $b$- and background (light- or in some cases $c$-jet) hypotheses, obtained from Monte Carlo simulation.  The IP3D tagger takes advantage of the signed transverse and longitudinal impact parameter significances. The SV1 tagger reconstructs an inclusive vertex formed by the decay products of the $b$-hadron and relies on the invariant mass of all tracks associated to the vertex, the ratio of the sum of the energies of the tracks in the vertex to the sum of the energies of all tracks in the jet and the number of two-track vertices. The JetFitter tagger exploits the topology of the primary, $b$- and $c$-vertices and combines vertex variables with the flight length significance.  The $b$-tagging performance is determined using a simulated $t\bar{t}$ sample and is calibrated using experimental data with jets containing muons and with a sample of $t\bar{t}$ events~\cite{ATLAS-CONF-2011-089}.


