%
%%%%%%%%%%%%%%%%%%%%%%%%%%%%%%%%%%%%%%%%%%%%%%%%%%%%%%%%%%%%%%%%%%%%%%%%%%%%%%%
% Object reconstruction and event selection
%%%%%%%%%%%%%%%%%%%%%%%%%%%%%%%%%%%%%%%%%%%%%%%%%%%%%%%%%%%%%%%%%%%%%%%%%%%%%%%
%

\chapter{Jet reconstruction and $b$-Tagging }\label{ch:reco}


This chapter describes the reconstruction of and physics contained within the hadronic final state in ATLAS.
Characteristics of the methods used for the identification of $b$-quark originated jet candidates and the properties of tracks in jets are also summarized. 

%------------------------------------------------------------------------
\section{Jets reconstruction and calibration}\label{sec:ObjSelection}
%------------------------------------------------------------------------

Four primary inputs to jet reconstruction are considered in ATLAS. Jets can be form from two different types of geometric calorimeter signal, calorimeter towers or topological clusters, or utilizing charged particle tracks reconstructed in the Inner Detectors. The calorimeter towers can combine all calorimeter cells in projective towers, or use only those with energies above a noise threshold. The latter are known as ``noise-suppressed'' towers.  The noise of a calorimeter cell is measured by recording calorimeter signals in periods where no beam is present in the acelerator.  The standard deviation $\sigma$ around the mean measured energy is interpreted as the noise of the cell, and dependes on the sampling layer in which the cell resides and the position in $\eta$.

The jets used in the present analysis are built from noise-suppressed topological clusters of energy in the calorimeter, aslo known as ``topo-clusters''. Topological clusters are three-dimensional objects seeded by a calorimeter cell with $|E_{cell}| > 4 \sigma$ above the noise. Neighbor cells that have an energy at least 2$\sigma$ above their mean noise are added to the cluster. In a final step, all nearest-neighbor cells surrounding the clustered cells are added to the cluster, regardless of signal-to-noise ratio. The position of the cluster is assigned as the energy-weighted centroid of all constituent cells.  

In either case, the jet inputs are combined as massless four-momentum objects in order to form the final four-momentum of the jet, which allows for a well-defined single-jet mass~\cite{Busato:1271710}.
%In the case of track-jets, the track four-momentum is constructed assuming the $\pi$ meson mass for each track.


In ATLAS, jets are reconstructed using two different size parameters: $R = 0.4$, for narrow jets, and $R = 0.6$, for wider jets. And the default jet algorithm is the anti-$kt$ algorithm~\cite{Asquith:1311867}. The effect of multiple simultaneous interactions on the reconstruction of jets is one of the primary factors that influence the decision on the jet algorithm, due to the expected level of pile-up in the LHC. The original ATLAS cone algorithm, known to contain infrared and collinear sensitivity, is too susceptible to this effect. On the contrary, the anti-$kt$ algorithm is the most stable after the introduction of pile-up.  


%FROM DAVID'S
%The jet constituents are uncalibrated; that is, the energy scale of the constituent is at the EM scale of the detector.
The baseline calibration of the calorimeters only corrects the energy measured in the detector to the EM energy scale. In order to compensate for the difference between the energy measurement of purely EM objects (electrons and photons) and the energy of a hadronic jet, an additional jet calibration procedure must be applied.  The procedure used for the 2011 data...

%FROM KIRSTEN'S
%something on reconstruction efficiency??
%The drop in efficiency at low pT is mainly due to jets that, as a result of jet energy resolution smearing, have energies below the 7 GeV threshold that is required by the ATLAS software in order to reconstruct jets.

The jet energy is first reconstructed from the constituent cell energies at EM-scale. These
cells have been calibrated to return the energy corresponding to electromagnetic showers
in the calorimeter, based on test-beam injection of electrons and pions [69], measurements
of cosmic muons [79], and reconstruction of the Z mass peak in Z → ee decays. Cells are
combined to form clusters, as described in Section 4.2, which are by construction massless
objects. The cluster four-momenta are then vectorially summed to yield the final jet four-
momentum, with its energy at EM-scale.
The purpose of the jet energy calibration, or jet energy scale, is to correct this mea-
sured EM-scale energy to the energy of the particles within a jet.
The jet energy calibration must account for energy lost in in-active regions of the de-
tector, such as in the cryostat walls or cabling; energy that escapes the calorimeters, such
as that of highly-energetic particles that “punch-through” to the muon system; energy of
cells that are not included in clusters, due to inefficiencies in the noise-suppression scheme;
and energy of clusters not included in the final reconstructed jet, due to inefficiencies in the
jet reconstruction algorithm. The muons and neutrinos that may be present within the jet
are not expected to interact within the calorimeters, and are not included in this energy
calibration.
Due to the varying calorimeter coverage, detector technology, and amount of upstream
in-active material, the calibration that must be applied to each jet to bring it to the hadronic
scale varies with its η position within the detector. 

%The jet response is defined EM in Monte Carlo as R = Ejet /Etruth , where each reconstructed jet at EM-scale is matched to a truth jet within ∆R < 0.3. 
The goal of the jet energy calibration is to bring the calibrated jet response to R = 1, flat
across all η.



%-------------------------------------------------------------------
\section{ $\bm b$-jet Tagging}\label{sec:btagging}
%------------------------------------------------------------------------


%From Gbb note
Jets are classified as $b$-quark candidates by the ATLAS MV1 $b$-tagging algorithm, based on a neural network that combines the information from three high-performance taggers: IP3D, SV1 and JetFitter \cite{ATLAS-CONF-2011-102}.  These three tagging algorithms use a likelihood ratio technique in which input variables are compared to smoothed normalized distributions for both the $b$- and background (light- or in some cases $c$-jet) hypotheses, obtained from Monte Carlo simulation.  The IP3D tagger takes advantage of the signed transverse and longitudinal impact parameter significances. The SV1 tagger reconstructs an inclusive vertex formed by the decay products of the $b$-hadron and relies on the invariant mass of all tracks associated to the vertex, the ratio of the sum of the energies of the tracks in the vertex to the sum of the energies of all tracks in the jet and the number of two-track vertices. The JetFitter tagger exploits the topology of the primary, $b$- and $c$-vertices and combines vertex variables with the flight length significance.  The $b$-tagging performance is determined using a simulated $t\bar{t}$ sample and is calibrated using experimental data with jets containing muons and with a sample of $t\bar{t}$ events~\cite{ATLAS-CONF-2011-089}.



%From https://atlas.web.cern.ch/Atlas/GROUPS/PHYSICS/CONFNOTES/ATLAS-CONF-2011-102/

The ability to identify jets containing $b$-hadrons is important for the high-$\pt$ physics program of a general-purpose experiment at the LHC such as ATLAS. Two robust $b$-tagging algorithms taking advantage of the impact parameter of tracks (JetProb) or reconstructing secondary vertices (SV0) have been quickly commisssioned~\cite{ATLAS-CONF-2010-091}\cite{ATLAS-CONF-2010-042} and used for several analyses of the 2012 and 2011 data (REFERENCIAS).
Building on this success, more advanced $b$-tagging algorithms have been commissioned with the 2011 data. All these algorithms are based on Monte Carlo predictions for the signal ($b$-jet) or background (light- or in some cases $c$-jet) hypotheses.

The $b$-tagging performance relies criticaly on the accurate reconstruction of the charged tracks in the ATLAS Inner Detector. 
The innermost part, the pixel detector, has an intrinsic measurement accuracy of around 10~$\mu$m in the transverse plane, and 115~$\mu$m along the beam axis ($z$).  For a central track with $\pt=$5~GeV, which is typical for $b$-tagging, the transverse momentum resolution is around 75~MeV and the transverse impact parameter resolutioni is about 35~$\mu$m.

\subsection{Primary vertex reconstruction}

The knowledge of the position of the primary interaction point (primary vertex) of the proton-proton collision is important for $b$-tagging since it defines the reference point with respect to which impact parameters and vertex displacements are measured.

See primary vertex recontruction in~\cite{ATLAS-CONF-2010-069}.

%\subsection{Pile-up}

Out-of-time pile-up events ($pp$ collisions from neighboring bunches in the same train) also generate calorimeter activity and consequently extra jets. However, given the time resolution of the Inner Detector, and since the $b$-tagging algorithms reject jets with no track associtaed to them, the contribution of the out-of-time pile-up for this analysis is expected to be negligible.




%------------------------------------------------------------------------
\subsection{Tracks selection and properties}\label{sec:ObjSelection}
%------------------------------------------------------------------------

\subsubsection{Track quality cuts}


The track selection for $b$-tagging is designed to select well-measured tracks rejecting fake tracks and tracks from long-lived particles ($K_s$, $\Lambda$, and other hyperon decays, generically referred to as $V^0$ decays) and material description.

The tracks of charged particles with a pseudorapidity $|\eta| < 2.5$ are reconstructed in the the Inner Detector. It is composed of a barrel, consisting of 3 Pixel layers, 4 double layers of single-sided silicon strip sensors, and 73 layers of Transition Radiation Tracker straws concentric with the beam, plus a system of disks on each end of the barrel, occupying in total a cylindrical volume around the interaction point of radius of 1.15~m and length of 7.024~m. The Pixel detector's innermost layer is located at a radius of 5 cm from the beam axis, has a position resolution of approximately 10~$\mu$m in the $r-\phi$ plane and 115 $\mu$m along the beam axis ($z$). %Tracks with $p^{\rm{track}}_{\rm{T}} > 150$
%Tracks with $p^{\rm{track}}_{\rm{T}} > 400$ MeV and consistent with the beamspot are associated in primary vertices via a finding/fitting sequential algorithm. Several primary vertices can be reconstructed per event due to the presence of in-time pile-up. The one with at least five associated tracks, a $z$ position within 100~mm of the ATLAS geometrical center, and the largest $\sum_{\rm trk}\pt^2$, is selected as the one associated to the hard interaction.
%are selected as good vertices.


\subsubsection{Track association to jets}

The actual tagging is performed on the sub-set of tracks in the event that are associated to the jets. Tracks are associtaed to the jets with a spatial matching in $\Delta R_{(jet,track)}$. The association cut $\Delta R$ is varied as a function of the jet $\pt$ in order to have a smaller cone for high-$pt$ jets which are more collimated.




\subsubsection{Impact parameters}
The most critical track parameters for $b$-tagging are the transverse and longitudinal impact parameters. The transverse parameter $d_0$ is the distance of closest approach of the track to the primary vertex point in the $r\phi$ projection. The $z$ coordinate of the track at this point of closest approach is referred to as $z_0$. It is often called the longitudinal impact parameter\footnote{Stricktly speaking the impact parameter is $|z_0|sin\theta$, where $\theta$ is the polar angle of the track.}. On the basis that the decay point of the $b$-hadron must lie along its flight path, the impact parameter is signed to further discriminate the tracks from $b$-hadron decays from tracks originating from the primary vertex. The sign is positive if the track extrapolation crosses the jet direction in front of the primary vertex, and negative otherwise. Therefore, tracks from $b/c$ hadron decays tend to have positive sign.

The significance, which gives more weight to tracks measured precisely, is the main ingredient of the tagging algorithms based on impact parameters.

\subsection{$b$-tagging algorithms}

%A first stag
