\%
%
%%%%%%%%%%%%%%%%%%%%%%%%%%%%%%%%%%%%%%%%%%%%%%%%%%%%%%%%%%%%%%%%%%%%%%%%%%%%%%%
% Theory
%%%%%%%%%%%%%%%%%%%%%%%%%%%%%%%%%%%%%%%%%%%%%%%%%%%%%%%%%%%%%%%%%%%%%%%%%%%%%%%
%
%\chapter{Quantum Chromodynamics}
\chapter{The theory of the strong interactions}

%------------------------------------------------------------------------
\section{The Standar Model}\label{sec:qcdintro}
%------------------------------------------------------------------------


%------------------------------------------------------------------------
%\section{QCD calculations}\label{sec:qcd}
%------------------------------------------------------------------------

%------------------------------------------------------------------------
\section{Jet physics}\label{sec:jets}
%------------------------------------------------------------------------

\subsection{Monte Carlo parton-shower programs}

%From Pythia's manual http://arxiv.org/abs/hep-ph/0603175
We have used a
perturbative language to describe the short-distance interactions of quarks, leptons and
gauge bosons.
For leptons and colourless bosons this language is sufficient. However, for
quarks and gluons it must be complemented with the structure of incoming hadrons, and a picture for the hadronization process, wherein the coloured partons are transformed into jets of colourless hadrons, photons and leptons.
The hadronization can be further sub-
divided into fragmentation and decays
This process is still not yet understood
from first principles, but has to be based on models.
In
the data, fluctuations arise from the quantum mechanics of the underlying theory. In
generators, Monte Carlo techniques are used to select all relevant variables according to the
desired probability distributions, and thereby ensure (quasi-)randomness in the final events.

QCD perturbation theory, formulated in terms of quarks and gluons, is valid at short
distances. At long distances, QCD becomes strongly interacting and perturbation the-
ory breaks down. In this confinement regime, the coloured partons are transformed into
colourless hadrons, a process called either hadronization or fragmentation. 


%From Gavins's http://arxiv.org/abs/1011.5131
Different classes of techniques can be used to make QCD predictions at hadron colliders, and in particular at the LHC. 
The so called Matrix Element Monte Carlos, including LO, NLO, and NNLO calculations. These ``fixed-order predictions'' involve the first couple of terms in the QCD perturbative expansion for a given cross section; as one includes further orders in the expansion one can reasonably hope to see systematic improvement in the accuracy of one's predictions.
%Are these characteristics representative of the ‘typical’ situation for collider observables? We only have predictions up to NNLO in a handful of cases (see below) and in those it is. In cases where wejust have NLO predictions, the features of large ‘K-factors’ (NLO/LO enhancements) with a reduced NLO uncertainty band are not uncommon, suggesting that beyond NLO corrections should be small. Exceptions are known to arise in two types of case: those where new enhanced partonic scattering channels open up at NLO (or beyond); and that involve two disparate physical scales.
%Technically, one main consideration has so far limited the range of processes for which NLO results exist: the availability of the loop amplitude. Until recently loop amplitudes were usually calculated semi-manually for each process. The complexity of the calculations increased significantly with the number of outgoing legs, limiting available results to those with at most three outgoing partons. Many NLO results for 2 → 2 and 2 → 3 processes are incorporated into programs such as NLOJ ET++ for jet production [65], MCFM for processes with heavy quarks and/or heavy electroweak bosons [66]

The Monte Carlo parton shower programs, which simulate the full hadronic final state, are crucial in evaluating detector acceptances and response.  
Knowing QCD predictions is crucial in the design of methods to search for new physics, as well as for extracting meaning from the data.

%how to get a parton shower event
%A quantity called Sudakov form factor (P (no emission above kt ) ≡ ∆(kt , Q)) allows us to easily calculate the distribution in transverse momentum of the gluon with largest transverse momentum in an event.
%FORMULA
%This distribution is easy to generate by Monte Carlo methods: take a random number r from a distribution that is uniform in the range 0 < r < 1 and find the kt1 that solves ∆(kt1 , Q) = r. Given kt1 we also need to generate the energy for the gluon, but that’s trivial. If we started from a q q system (with some randomly generated orientation), then this gives us a q q g system. As a next step one can work out the Sudakov form factor in the soft/collinear limit for there to be no emission from the q q g system as a whole above some scale kt2 (< kt1 ) and use this to generate a second gluon. The procedure is then repeated over and over again until you find that the next gluon you would generate is below some non-perturbative cutoff scale Q0 , at which point you stop. This gives you one ‘parton shower’ event.
%The above shower descriptions hold for final-state branching. With initial-state hadrons, one also needs to be careful with the treatment of the PDFs, since the collinear splitting that is accounted for in the parton shower is also connected with the way the PDF is built up at the scale of the hard scattering.
(This) procedure for generating a parton shower event is present in PYTHIA (pt-oredered and HERWIG (angular ordering), the ones we use in this thesis and others like SHERPA. (REFERENCIAS!). 

Hadronization cannot be calculated from first principles, but has to be
modelled. The two most commonly used model classes are the string
and cluster ones.

Real events consist not of partons but of hadrons. Since we have no idea how to calculate the transition between partons and hadrons, Monte Carlo event generators resort to ‘hadronization’ models. One widely-used model involves stretching a colour ‘string’ across quarks and gluons, and breaking it up into hadrons [REFERENCIAS]. This is the Lund model  in Monte Carlo program PYTHIA.
Another model breaks each gluon into a q q pair and then groups quarks and anti-quarks into colourless ‘clusters’, which then give the hadrons. This cluster type hadronization is implemented in the HERWIG event generator [REFERENCIAS] and recent versions of SHERPA.
Hadronization models involve a number of ‘non-perturbative’ parameters. The parton-shower
itself involves the non-perturbative cutoff Q0 . These different parameters are usually tuned to data from
the LEP experiments.
In addition to the hard interaction that is generated by the Monte Carlo
simulation, it is also necessary to account for the interactions between the incoming proton 
remnants. This is usually modelled through multiple extra 2 → 2 scattering, occurring at a scale of a
few GeV, and known as multiple parton interactions. This modelling of the underlying event is crucial in
order to give an accurate reproduction of the (quite noisy) energy flow that accompanies hard scatterings
in hadron-collider events.




\subsubsection{Pythia}

PYTHIA is a general-purpuse event generator. It has been used extensively for $e^+ e^-$, $ep$ and $pp/p\bar{p}$, e.g at LEP, HERA and the Tevatron, and during the last 20 years has probably been the most used generators for LHC physics studies. 

%PYTHIA in its version 6 is the product of over thirty years of progress. 

PYTHIA contains an extensive list of hardcoded subprocesses, over 200, that can be switched on individually. These are mainly 2$\rightarrow$1 and  2$\rightarrow$2, some  2$\rightarrow$3, but no multiplicities higher than that. Consecutive resonance decays may of course lead to more final-state particles, as will parton showers.

In PYTHIA generator showers are ordered in transverse momentum (REFERENCIA)%http://arxiv.org/abs/hep-ph/0408302
both for ISR and for FSR. Also MPIs are ordered in $\pt$.
%Parton showering
The initial- and final-state algorithms are partly based on a dipole-type approach to recoils.

%Multiple parotn interactions and beam remnants
MPI modelling has traditionally been a hallmark of PYTHIA. 

%Hadronization 
Hadronization is based solely on the Lund string fragmentation framework.
%the Field–Feynman model http://www.sciencedirect.com/science/article/pii/0550321378900159 (sent pdf by email, only accesible from cern)
% a few years later kickstarted the whole field of hadronization studies by Monte Carlo simulation
%On the Lund's String model
%Now consider a simple qqbar two-parton event further. As thea q and qbar move apart from the creation vertex, say along the ±z axis, the potential energy stored in the string increases, and the string may break by the production of a new qprima qbarprima pair, so that the system splits into two colour-singlet systems qqbarprima and qprima qbar. These two systems move apart, and a widening no-field region opens up in between, Fig. 13b. For simplicity the quarks are shown as massless, so they move with the speed of light. If the invariant mass of either of these systems is large enough, further breaks may occur, and so on until only ordinary hadrons remain. Typically, a break occurs when the q and the qbar ends of a colour singlet system are 1–5 fm apart in the qqbar rest frame, but note that the higher-momentum particles at the outskirts of the system are appreciably Lorentz contracted.

%From the GbbNOte
Samples of dijet events from proton-proton collision processes were simulated with P YTHIA 6.423
[11]%T. Sjostrand, S. Mrenna, and P. Skands, PYTHIA 6.4 Physics and Manual, JHEP 05 (2006) 026,arXiv:hep-ph/0603175
 using a 2 → 2 matrix element at leading order in the strong coupling to model the hard subprocess,
pT -ordered parton showers to model additional radiation [12]%R. Corke and T. Sj ̈ strand, Improved parton showers at large transverse momenta, Eur.Phys.J.C 69 (2010) 1, arXiv:1003.2384 [hep-ph].
, underlying event and multiple parton interactions [13]%T. Sj ̈ strand and P. Z. Skands, Transverse-momentum-ordered showers and interleaved multiple interactions, Eur.Phys.J.C 39 (2005) 129, 0408302 [hep-ph].
, and fragmentation and hadronisation based on the Lund string model [14]%B. Andersson et al., Parton fragmentation and string dynamics, Phys. Rep. 97 (1983) 31. 
The ATLAS
AMBT2 tune of the soft model parameters was used [15]%ATLAS Collaboration, ATLAS tunes of PYTHIA 6 and Pythia 8 for MC11,ATL-PHYS-PUB-2011-009 (2011) .





\subsubsection{Herwig$++$}

Herwig$++$ is based on the event generator HERWIG (Hadron Emission Reactions With Interfering Gluons), which was first published in 1986 and was developed throughout the era of LEP.  HERWIG was written in Fortran, and the new generator, Herwig$++$ developped in C$++$. The version used in this thesis is versin 2.4.2 released in 2009. Some distinctive features of Herwig$++$ are:

\begin{itemize}\addtolength{\itemsep}{-0.4\baselineskip}
\item
Angular ordered parton showers.
\item
Cluster hadronization.
%The cluster model of hadronization is based on the so-called preconfinement property of parton showers, discovered by Amati and Veneziano http://www.sciencedirect.com/science/article/pii/0370269379908967
%They showed that the colour structure of the shower at any evolution scale Q0 is such that colour singlet combinations of partons (clusters) can be formed with an asymptotically universal invariant mass distribution. Here ‘universal’ means dependent only on Q0 and the QCD scale Λ, and not on the scale Q or nature of the hard process initiating the shower, while ‘asymptotically’ means Q Q0 . If in addition Q0 Λ, then the mass distribution of these colour singlet clusters, together with their (Q-dependent) momentum and multiplicity distributions, can be computed perturbatively.
%Once these clusters have been formed, how should they decay into the observed hadrons? The typical cluster masses are low enough for them to be treated as a smoothed-out spectrum of excited mesons, in which case quasi two-body decay into less excited states seems to be preferred by Nature.

\item 
Hard and soft multiple partonic interactions to model the underlying event and soft inclusive interactions (REFERENCIA) %http://arxiv.org/abs/0803.3633
%\item 
\end{itemize}


A more detailed descripcion can be found elsewhere (REFERENCIAS).
%http://arxiv.org/abs/0803.0883
%http://arxiv.org/abs/arXiv:1101.2599

\subsection{Jet algorithms}

%------------------------------------------------------------------------
\subsection{Jet substructure}
%------------------------------------------------------------------------
