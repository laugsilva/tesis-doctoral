%%%%%%%%%%%%%%%%%%%%%%%%%%%%%%%%%%%%%%%%%%%%%%%%%%%%%%%%%%%%%%%%%%%%%%%%%%%%%%%
% Itroduccion
%%%%%%%%%%%%%%%%%%%%%%%%%%%%%%%%%%%%%%%%%%%%%%%%%%%%%%%%%%%%%%%%%%%%%%%%%%%%%%%

\chapter{Introduction}

The first years of proton-proton collisions at a centre of mass energy of 7~TeV delivered by the Large Hadron Collider and recorded by the ATLAS experiment have provided data to
explore quantum chromodynamics (QCD) at scales never reached before. Precision measurements of strong interactions are interesting in their own right, but, in addition, QCD provides one of the main
backgrounds to many New Physics measurements; furthermore, it is also through tests of QCD
that New Physics may be discovered.

Due to QCD confinement the experimental signature of quarks and gluons are not the quarks and gluons themselves but a spray of ``colorless'' hadrons, that we call \emph{jets}.  Hadronic jets are a fundamental ingredient for precision tests of QCD: understanding and measuring their performance is crucial in the LHC environment. A wide range of physics signatures, within the Standard Model (SM) and Beyond the Standard Model (BSM) predictions, contain jets originating from bottom ($b$) quarks. 
%Within the SM a range of production channels exist for heavy-quark jets, $\eg$ pure QCD production or in association with heavy bosons ($W, Z, H$). Furthermore, $b$-quarks enter in many collider searches, notably because they are produced in the decays of various SM particles, \eg top quarks and the Higgs boson, and of numerous particles appearing in proposed extensions of the SM. 
The ability to identify jets containing $b$-hadrons, the product of the hadronization of $b$-quarks, is therefore important for the high-$\pt$ physics program of the ATLAS experiment. 

$b$-tagging algorithms rely on the relatively long decay length of $b$-hadrons that gives rise to large impact parameter tracks and displaced decay secondary vertices; or on the presence of a soft lepton within the jet, the product of the semileptonic $b$-decay.   
These algorithms, however, do not provide information on the number of $b$-hadrons within the jet. In particular, they tag  ``merged'' jets containing a $b\bar{b}$ pair, with no net heavy flavour, which do not correspond to the intuitive picture of a $b$-jet as a jet containing a single $b$-quark or antiquark.
%In particular they tag gluon jets if they give rise to a close-by $b$-hadron pair via gluon splitting. 
%, as depicted in Fig.~\ref{fig:gbbcartoon}. 

Successfully tagging merged $b$-jets, which in QCD are produced mainly from gluon splitting $g \rightarrow b\bar{b}$, is important to reduce and to improve the estimation of the $b$-tag background to Standard Model analyses and to new physics searches involving b-jets in the final state.  In particular, it has been shown that efficient tagging of gluon splitting jets can also help in reducing the theoretical uncertainties in the calculation of the inclusive $b$-jet spectrum~\cite{Salam.AccurateHQ}.










%%%%%%%%%%%%%%%%%%%%%%%%%%%%%%%%%%%%%%%%%%%%%%%%%%%%%%%%%%%%%%%%%%%
%  Our work, plus content
%%%%%%%%%%%%%%%%%%%%%%%%%%%%%%%%%%%%%%%%%%%%%%%%%%%%%%%%%%%%%%%%%%%



%At the Tevatron acelerator (Fermilab) 50\% of the $b$-hadrons are due to the gluon splitting process; a larger fraction is expected to contribute at the LHC.
%\vspace{3mm}
There are two possible strategies to attempt to identify $b$-jets containing two $b$-hadrons in hadronic collisions. One of them, implemented at the CDF experiment at Fermilab~\cite{CDFAzimutalCorrelation}, relies on the direct reconstruction of the two $b$-decay secondary vertices. This %has the further advantage of allowing 
allows the measurement of the angular separation between the $b$-hadrons, but suffers from the low efficiency of a double $b$-tag requirement plus additional reconstruction inefficiencies at small angular separation between the two $b$-hadrons. In this thesis we develop for the first time an alternative method that does not rely on explicit vertex finding, but exploits the substructure differences between single and merged $b$-jets, combining them in a multivariate analysis. 
The method developed is then applied to measure the fraction of double $b$-hadron jets as a function of jet $\pt$, using 4.7 fb$^{-1}$ of $pp$ collision data at $\sqrt{s}=7$~TeV collected by the ATLAS experiment in 2011.

The thesis is organized as follows: Chapter~\ref{ch:theory} describes the theoretical framework, with emphasis in the theory of the strong interactions and the aspects that are important for the understanding of the hadronic final state in hadronic collisions. The LHC and the ATLAS detector are described in Chapter~\ref{ch:lhc_atlas}, together with a summary of the experimental conditions during the 2011 data taking.  Chapter~\ref{ch:reco} details how jet reconstruction and calibration are performed at ATLAS and describes the procedure for the identification of $b$-quark jets. Chapter~\ref{ch:kinematic} presents the analysis of jet shape and substructure variables for the discrimination between single and double $b$-hadron jets. The validation of the variables in 2011 data is also included.   The construction of the multivariate discriminator  and the discussion of its systematic uncertainties are presented in Chapter~\ref{ch:mva}. 
Chapter~\ref{ch:gbbfraction} details the technique used for the measurement of the fraction of double $b$-hadron jets in QCD $b$-production and the associated systematic uncertainties.
Finally, chapter~\ref{ch:conclusions} presents a summary and conclusion. 




