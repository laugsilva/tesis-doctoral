\documentstyle[12pt,epsf,esp]{dfletter}
\setlength{\textheight}{220mm}
\setlength{\topmargin}{15mm}
\setlength{\headheight}{0cm}
\setlength{\headsep}{0cm}
\setlength{\textwidth}{160mm}


\date{Noviembre 13, 2012}
\Email{aia@df.uba.ar}
\begin{document}
 
%\newcommand{\'i}{\mbox{\' \i}}
\newcommand{\kt}{k$_\perp$\ }
\def\D0{D\O}
\def \GeV {\rm GeV}
\def\etal{{\sl et al.}}


\begin{letter}{\ }
 
\vspace{-6cm}
\opening{Sres. Miembros de la \\
        Comisi\'on de Doctorado \\
        Facultad de Ciencias Exactas y Naturales \\
        Universidad de Buenos Aires}

\vspace{5mm}

{\large \bf Evaluaci\'on Acad\'emica - Lic.\ Mar\'ia Laura Gonz\'alez Silva }

\vspace{5mm}

Me dirijo a Uds con el fin de elevar la evaluaci\'on acad\'emica de la Lic.\
Mar\'ia Laura Gonz\'alez Silva, Tesista de Doctorado en el Departamento de
F\'isica de esta Casa de Estudios, quien presentar\'a su Tesis a la brevedad.

El trabajo de Laura se desarroll\'o en el marco del experimento ATLAS del Gran
Colisionador de Hadrones (LHC), Laboratorio CERN, Ginebra, Suiza, una
colaboraci\'on de aproximadamente dos mil miembros que estudia diversas ramas
de la F\'isica de Altas Energ\'ias a trav\'es de la producci\'on de
part\'iculas en colisiones prot\'on-prot\'on a 7~TeV en el centro de masa. El
trabajo dentro de ATLAS se puede dividir en dos grandes categor\'ias. Por un
lado las tareas de infraestructura, como es por ejemplo la simulaci\'on y
calibraci\'on de los detectores, el desarrollo de algoritmos de disparo, o la
reconstrucci\'on de eventos.  Por otro lado se encuentran las l\'ineas de
an\'alisis f\'isico propiamente dichas, que est\'an divididas en seis grandes
\'areas: Modelo Est\'andar, Medici\'on de propiedades del quark top, Bos\'on de
Higgs, B\'usqueda de Supersimetr\'ia, Procesos ex\'oticos, e Iones pesados. Los
alumnos de doctorado deben tomar un rol activo en las tareas de inter\'es
general y adem\'as ser responsables de una l\'inea de an\'alisis.

Laura tuvo un rol importante en un conjuntos tarea de infraestructura: el
desarrollo del disparador para muones c\'osmicos, que permiti\'o poner a punto
el detector y el sistema de toma de datos antes de las primeras colisiones
hadr\'onicas, la mejora de la resoluci\'on en energ\'ia de los jets utilizando
informaci\'on de trazas en adici\'on a la calorim\'etrica, el estudio de la
topolog\'ia de celdas ocupadas por jets en el calor\'imetro hadr\'onico, el
desarrollo de un software de metadata para la selecci\'on %clasificaci\'on 
de eventos de
$b$-tagging para an\'alisis f\'isicos, y la determinaci\'on de la respuesta de
jets en eventos top, que est\'an dominados por jets de quarks a diferencia de
eventos QCD donde dominan los jets de gluones.

En lo que respecta al an\'alisis f\'isico, Laura dise\~n\'o y desarroll\'o un
algoritmo de identificaci\'on de jets que contienen quarks $b$ pero que
provienen del desdoblamiento de un glu\'on. Esta distinci\'on es muy importante
pues permite separar jets glu\'onicos de QCD (en los que el glu\'on se desdobla en
un par quark-antiquark $b$) de aquellos jets-$b$ genuinos, como los producidos
por el decaimiento del quark top, del bos\'on $W$ y por distintos modelos de
f\'isica m\'as all\'a del Modelo Est\'andar. Este desarrollo requiri\'o un
estudio de reciente ideas t\'eoricas, la concepci\'on y optimizaci\'on del
algoritmo, la determinaci\'on de su eficiencia, y su aplicaci\'on a la
medici\'on de la fracci\'on de jets-$b$ producidos por gluones en datos. Esta
es adem\'as una herramienta que ser\'a utilizada por distintos an\'alisis
f\'isicos que dependen de la presencia de quarks $b$ y donde el desdoblamiento
de gluones es una contaminaci\'on a la se\~nal estudiada.

En reconocimiento por este desarrollo el experimento ATLAS la design\'o
oficialmente representante de la Colaboraci\'on como oradora en la conferencia
BOOST 2012, que tuvo lugar en Valencia, Espa\~na. Entre las publicaciones que
ha dado lugar el trabajo de tesis Laura cabe resaltar las siguientes:

\begin{itemize}

\item ``Commissioning of the ATLAS Muon Spectrometer with Cosmic Rays'', ATLAS Collaboration, Eur.Phys.J. C70 875 (2010)

\item ``Studies of the performance of the ATLAS detector using cosmic-ray muons'', ATLAS Collaboration, Eur.Phys.J. C71 1593 (2011)

\item ``Measurement of inclusive jet and dijet cross sections in proton-proton collisions at 7 TeV centre-of-mass energy with the {ATLAS} detector'', ATLAS Collaboration, Eur.Phys.J. C71 1512 (2011)

\item ``Jet energy measurement with the ATLAS detector in proton-proton collisions at $\sqrt{s}$ = 7 TeV'', ATLAS Collaboration, arXiv:1112.6426, enviado para su publicaci\'on a Eur.Phys.J. C

\item ``Identification and Tagging of Double $b$-hadron jets with the ATLAS Detector'', ATLAS Collaboration Public Note, https://cdsweb.cern.ch/record/1462603

\end{itemize}

Atentamente,

\vspace{5mm}

\begin{flushleft}
Dr. Ricardo Piegaia 
\end{flushleft}

\end{letter}
\end{document}
